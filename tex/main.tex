\environment l10n-ar

\setvariables
  [l10n-manual]
  [title=إحداث التغيير بتوطين المعلوماتية]

\starttext

\startfrontmatter
\section{مقدمة}
الحواسيب أدوات رائعة يمكن الانتفاع بها على أوجه لا حصر لها، إلا أننا
ننزعج عندما لا تعمل على النحو الذي نريد. نتفاعل باستمرار مع التقنية
فنغير من طريقة عملها وتغير هي بدورها من طريقة عملنا، ومع أن هذا التغيير
كثيرا ما يكون محمودا، إلا أننا لا نكون دوما واعين لكيف يغيرنا.

تحتوي كثير من التقنيات التي نتعامل معها –مثل الحواسيب ومواقع الوب
والهواتف المحمولة– على نصوص بلغة ما، وهي غالبا إحدى اللغات الأكثر
انتشارا في العالم مثل الإنگليزية أو الفرنسية، مما يقصر إمكانية استخدام
تلك التقنيات على من يفهمون تلك اللغات، ويحرم غيرهم من استخدام التقنية
أو الانتفاع بكل فوائدها.

يشرح هذا الكتاب أساليب {\it توطين} البرمجيات؛ أي ترجمتها وتطويعها
لاحتياجات المستخدمين. يشرح الكتاب أدوات تستخدم للترجمة و مسائل متعلقة
بترجمة البرمجيات إلى لغات أخرى. ترجمة البرمجيات تقرِّب التقنية إلى
المتحدثين بتلك اللغة وتتيح استعمال لغات أكثر في تقنيات المعلوماتية.

من المهم الوعي بأهمية توطين البرمجيات، فحتى وإن استطعنا استخدام التقنية
بلغة أجنبية فإن التعرض المستمر للغات الأجنبية يُضعف تدريجيا من ملكة
اللغة الأم. و نحن نعتقد أن اللغات التي لا تُستعمل في كل النواحي الهامة
للحياة اليومية تواجه خطر الانقراض في عصر العولمة الطاغية. لذا فتوطين
البرمجيات لا يساعد فقط على إتاحة التقنية للعامة، بل ويساعد أيضا على
إحياء اللغات والمحافظة على ثقافاتها.

يُعنى هذا الكتاب على وجه الخصوص بتوطين ما يعرف {\it بالبرمجيات الحرة
مفتوحة المصدر}، وهي برمجيات يمكن توطينها ونشرها بحرية، والكثير من هذه
المشروعات البرمجية لها الريادة في دعم اللغات المختلفة، مع هذا فالكتاب
يبقى مفيدا في مجال توطين البرمجيات عموما، وفي المساهمة في البرمجيات
الحرة في مجالات أخرى غير التوطين.

بالرغم من تعدد الموضوعات التي يعرض لها هذا الكتاب إلا أنه لا يشمل كل شيء
بهدف إبقائه في حجم معقول. توجد موضوعات أخرى عديدة تتعلق بلغة بعينها أو
أداة بعينها، غير أننا في هذا الدليل نقتصر على التعرض للموضوعات
المشترَكة في توطين أغلب البرمجيات الحرة، وعلى أداتي بُوتِل وفِرتال.

\page

\section{إشادات}
بعض محتويات هذا الكتاب بُنَيَت على مواد من موقع مشروع Translate\rlm:
\CURL{http://translate.sourceforge.net/}

نشكر صمويل مُري على مقالته التي استعنّا بها في قسم المصطلحات. المقالة
الأصلية باللغة الأفريقانية:
\CURL{http://leuce.com/translate/termhelp.html}

نشكر أحمد غربية و خالد حسني على توصيات تخص اللغة العربية و على الإشارة
إلى موقع عربآيز \URL{http://wiki.arabeyes.org} كمصدر معلومات للموطنين العرب.

\section{الرُّخصة}
هذا الكتاب مُرخَّصٌ برخصة المشاع الإبداعي: النِّسبة، غيرالتجاري، الترخيص
بالمثل:
\CURL{http://creativecommons.org/licenses/by-nc-sa/3.0}

\completecontent

\stopfrontmatter

\startbodymatter

\chapter{تمهيد}
\section{توطين البرمجيات}
تُصمَّم البرمجيات عادة بحيث تتواصل مع المستخدِم بلغة ما مثل الانجليزية،
لذا تنبغي ترجمة الرسائل التي تعرضها ليمكن لمتحدثي اللغات الأخرى
استخدامها، وهذا هو لب صيرورة {\it توطين البرمجيات}. ويُقصد بالتوطين كل
التغييرات التي تُجرى على المُنتَج لتطويعه إلى لغة أو بلد أو ثقافة أو
حتى نظام قانوني مختلف.

\placefigure
  [here,force,nonumber]
  {انظر كيف طُوِّعَ محرر النصوص هذا ليناسب اللغة العربية}
  {\externalfigure[img001.png][width=15.108cm]}

مع أن الترجمة هي الجزء الأهم في عملية التوطين إلا أن تغييرات أخرى يجب
أخذها في الحسبان لتطويع برمجية لمجموعة من المستخدمين. فقد تكون حاجة
لاستبدال بعض الصور أو إشارات العملة المحلية أو تنسيق الأرقام والتاريخ
حسب الطريقة المتبعة في اللغة المستهدفة.

كثيرا ما تحسب شركاتُ البرمجياتِ التوطينَ وسيلة لفتح أسواق جديدة في بلدان
لم يسبق لها العمل فيها، والتوطين صناعة كبيرة. فبينما لم تكن حتى أشهر
البرمجيات في التسعينيات توجد سوى ببضعة لغات صار من المعتاد اليوم أن
تتوفر بعض البرمجيات بأكثر من ثلاثين لغة، بل وأكثر من خمسين
أحيانا.\footnote{صدرت الإصدارة 3,6 من موزيلا فَيَرفُكس في خمس وستين لغة
إضافة إلى لغات أخرى تتوفر عبر إضافات للبرمجية.}

تمهد إتاحة البرمجيات بلغة جديدة طريق استخدامها أمام طائفة من المستخدِمين
ربما لم تكن تستطيع استغلالها من قبل، وقد لا يستطيع بعض الناس الانخراط
في العالم الرقمي دون البرمجيات المُوطّنة. ومع التوسع في استغلال
الحواسيب لإنجاز مهام أكثر وأكثر تزيد الحاجة إلى إتاحتها لكل الناس حتى
لمن لا يجيدون غير لغة واحدة. توطين البرمجيات يتيح للمستخدمين أن يكونوا
مواطنين من الدرجة الأولى في العالم الرقمي دون أن تعوقهم حواجز اللغة أو
أن يشعروا أن لغتهم غير هامّة. استعمالك لغتك الأم في عالم تقنية
المعلوماتية من مقوّمات الحفاظ عليها و إبقاءها مفيدة في الحياة اليومية.

\section{توطين البرمجيات الحرة}
مع أن كثيرا من الشركات تبيع برمجيات الحاسوب أو تعطيك ترخيصا وحسب
باستخدامها مقيدا، إلا أنه توجد برمجيات كثيرة دون هذه القيود بل وكثيرا
ما تكون مجانية. يُطلق على هذه البرمجيات {\it البرمجيات الحُرَّة} أو
{\it البرمجيات مفتوحة المصدر}، ويُولي مناصروها اهتماما كبيرا {\it
للحريات} الممنوحة لمستخدمي هذه البرمجيات و لتوافرها لكل الأغراض. توطين
هذه البرمجيات هو محور هذا الكتاب.

لمطالعة المزيد عن البرمجيات الحرة في:
\URL{http://www.gnu.org/philosophy/free-sw.html}

وللمزيد عن البرمجيات مفتوحة المصدر طالع:
\URL{http://opensource.org/docs/osd}

البرمجيات التي تخضع لتراخيص أكثر تقييدا بحيث لا تكون حرة أو مفتوحة
المصدر كثيرا ما تسمى {\it برمجيات مملوكة}. ومع أن بعض البرمجيات
المملوكة موجهة للسوق العالمي وتتوفر بالعديد من اللغات إلا أن توطين تلك
البرمجيات في يد مطوريها وإليهم يرجع القرار في أي اللغات يدعمونها ومتى
تصدر النسخ المترجمة.

لكن الأمر مختلف في حال البرمجيات الحرة إذ يُمكن لأيٍ كان تطويعها لتلائم
احتياجاته، فإذا كنا نؤمن بأهمية توطين برمجية ما، فإن البرمجية الحرة
يُمكننا توطينها ونشر النسخ المعدلة، وعادة ما يقوم على تطوير البرمجيات
الحرة مجتمع يهمه نجاحها لذا يرحب بمساهمة المترجمين بالترجمة والتحسينات
الأخرى الضرورية لتناسب البرمجيات الحرة جمهورا أوسع. 

ولأن البرمجيات الحُرَّة تعطينا الحرية لنفعل بها المزيد ولأنها تتوفر في
الأغلب بلا مقابل أو بتكلفة زهيدة، فقد يكون التطوع بتوطين تلك البرمجيات
ردًا للجميل و مساهمة إيجابية في المجتمع.

لهذا تتعدد أسباب توطين البرمجيات الحرة:

\startitemize[1]
\item الجميع بوسعه المشاركة والمساهمة بالتعديلات التي تتطلبها لغتهم
\item مساهمة في نشر البرمجيات التي تقدّر حريتك
\item دفع مطوري البرمجيات المملوكة إلى تحسين برمجياتهم بطريق منافستهم
\item التمرُّس على توطين وترجمة البرمجيات
\stopitemize

\section{الاختلافات عن الترجمة العادية}
أي مترجم متمرس سيخبرك أن الترجمة ليست عملية آلية، بل هي فن وموضوع
للأبحاث والدراسة الأكاديمية. وعندما نرى قصور برمجيات الترجمة الآلية
فإننا نعي أهمية الاهتمام بالتفاصيل الدقيقة لتخرج الترجمة مقبولة. ومبادئ
ترجمة الوثائق تنطبق كذلك على توطين البرمجيات، و كُلُّ خبرة مكتسبة في
الترجمة تفيد في الترجمة، و لا تظُنَّ أن كل مبرمج يستطيع ترجمة البرمجيات
لمجرد معرفته بالحوسبة.

لكن توجد اختلافات بين ترجمة البرمجيات وباقي أنواع الترجمة، منها الطبيعة
التقنية للنصوص المترجمة لكنها ليست الاختلاف الوحيد، فالأدوات المستخدمة
في الترجمة مختلفة، كما تختلف القيود على ما يُمكن فعله. خصصنا فصلا كاملا
في هذا الكتاب لمناقشة المسائل التقنية التي تظهر عند توطين البرمجيات
(انظر صفحة \at[ref:32352020]). لذا ينبغي الأخذ في الحسبان أن
حتى أمهر المترجمين بحاجة إلى التمرس على المسائل التي تختص بها ترجمة
البرمجيات.

من المهم الانتباه إلى أن الوثيقة المترجَمة عادة تقرأ بتتابع معروف من
أولها إلى آخرها على الصورة التي وضعها المترجم عليها، لكن في حال
البرمجيات لا تمكن تحديد ما سيراه المستخدم و بأي ترتيب، إذ تختلف طريقة
استخدام البرمجية من شخص لآخر لذا فقد يطالع مستخدِم ما المترجمة بطرق غير
الذي تراها أنت عندما تترجمها.

\section{الاختلافات عن التوطين التجاري}
إن كانت لك خبرة بتوطين البرمجيات المملوكة، فإنك على الأغلب لاحظت
الاختلاف بينه و توطين البرمجيات الحرة. فوجود المال أو عدمه قد يكون له
أثر على حفز المساهمة، و قد تعني ترجمة البرمجيات الحرة مزيدا من الحرية
للمترجم، كما قد لا يفرض المشروع دورة عمل معينة.

عادة ما يكون التواصل مباشرا بين المبرمجين والمترجمين في المشروعات الحرة،
مما يسرّع الإجابة عن الاستفسارات أو المشكلات المتعلقة بالنص المصدر.
كذلك فالمستخدمون والمترجمون يكونون أكثر قربا مما يتيح تشاورا مباشرا.

كثير من المشروعات الحرة ترحّب بالمُوَطِّنين و يسودها إدراك بأهمية جذب
المتطوعين لذا يعمل أفرادها بجد لتسهيل مشاركتهم، ولهذا السبب كثيرا ما
تتلقى المشروعات الحرة المشهورة مساهمات ترجمة بلغات كثيرة، بل وأحيانا
بلغات لا تجذب الاهتمام التجاري خارج عالم البرمجيات الحرة.

فيما يخص للأدوات يُفضل بالطبع استخدام أدوات حرة لتوطين البرمجيات الحرة.
نَسَق الملفات PO هو الأكثر شيوعا، ولأنه يحوى النص المصدر إضافة إلى
الترجمة (مثله مثل XLIFF) فالاهتمام أقل بإعادة الترجمة وتبادل ذواكر
الترجمة. قسم “صيرورة توطين بسيطة” في صفحة \at[ref:38565525]
يعرض مثالا على صيرورة التوطين.

\chapter{نظرة عامة}
\section{ما الذي تسعى لتحقيقه؟}
\subsection{تحديد الأهداف}
في الفصل السابق رأينا دوافع عديدة لتوطين البرمجيات الحرة، فدافعك قد
يختلف عن المشاركين الآخرين، لكن هذه ليست مشكلة في الأغلب، كل ما عليك هو
أن تضع نصب عينيك هذا السؤال:

{\it
ما الذي تسعى لتحقيقه؟}

إن كنت تريد توطين البرمجيات لمجرد المتعة فغالبا لا يهمك كثيرا أي برمجية
تترجم ولا كيف. لكن كثيرا من الناس يسعون وراء هدف معين، فقد يكون هدفك
مساعدة الناس في بلدك، أو المحافظة على اللغة، أو ربما دعما لهذه البرمجية
أو تلك. كلها أهداف نبيلة، لكن يجب تحديدها بوضوح ووضعها دائما نصب عينيك
لأن كثيرا من القرارات التي ستأخذها تنبني على الهدف.

سواء كنت تعمل مستقلا أو في مجموعة أو حتى في مؤسسة رسمية، فسيفيدك تحديد
أهداف واضحة في التركيز على ما تسعى إليه. كثير من الشركات تعمل على وضع
رؤية وهدف محدد أثناء مرحلة التخطيط، فمعرفة ما تسعى إلى تحقيقه لا يفيد
وحسب في إبقاء هدفك نصب عينيك، بل يسهِّل تحديد معيار للنجاح.

إذا كنت تعمل متطوعا فعليك أن تحاول جعل الأمر ممتعا، اجعل هذا هدفا ثانويا
إذا أردت.

\subsection{تحديد الأهداف لتحقيق المقاصد}
في مشروع Translate.org.za في جنوب أفريقيا أردنا أن تمكين الناس و دعم
لغاتهم، أو هكذا انتوينا، لكننا في البداية أخطأنا الطريق الأمثل لتحقيق
هذا الهدف، فكنا نترجم البرمجيات التي تعجبنا بدلا من البرمجيات التي قد
يستخدمها الناس أكثر. ولأننا لا نستطيع ترجمة كل شيء كان علينا الاختيار
وتحديد الأولويات. كلما حددت أهدافك بدقة استطعت تحديد أولويات المشروعات
المختلفة و الوصول لأهدافك أسرع أو بنجاح أكبر.

محدودية الموارد المتاحة لك (سواء المتطوعين أو المال أو الوقت إلى غير
ذلك) تُحتِّم التخطيط لوسائل استدامة واستمرارية المشروع. فكِّر في كيفية
تحفيز المتطوعين. و إن توفر المال سيكون عليك وضع ميزانية ومراقبة مدى
الالتزام بها، وعليك كذلك حساب الوقت المستغرق في إنجاز العمل لتستطيع
تنظيم الوقت أفضل. قسم “تحديد الأولويات (عندما لا تستطيع فعل كل شيء)”
صفحة \at[ref:36383525] يناقش بعض وسائل تقييم المشروعات وتحديد
أولوياتها.

\section{مبادئ ينبغي الانتباه لها}
يتطلب التوطين تعلُّم الكثير، وهذا قد يجعلها تبدو مهمة صعبة، مع هذا فمن
الممكن إخراج ترجمة جيدة بتفادي بعض الأخطاء المهمة، و كذلك الاستمتاع
أثناء التعلم. وهذه بعض المبادئ التي ينبغي الانتباه لها.

{\bi ابدأ بداية متواضعة.} ليس عليك البدء بأكبر البرمجيات وأكثرها شعبية،
اختر مهمة يمكنك أداؤها بسهولة في الوقت المتاح لك، إتمام المهمة سيُشعرك
بالفخر ويحفزك أكثر.

{\bi اقرأ التعليمات.} فغالبا ستجد بها نصائح تفيد في حالات معينة. فقد
ينشر مشروع البرمجية التي تترجمها تعليمات للمساعدة في الأجزاء الصعبة، أو
قد يضع المبرمج تعليقات على جمل معينة لشرح مفاهيم غير واضحة. الاستعانة
بدليل للقواعد اللغوية يساعد في تفادي الأخطاء الشائعة ومشاكل النحو أو
الإملاء التي قد تواجهك.

{\bi حافظ على الاتساق.} الاتساق أحد الأهداف المهمة في التوطين. يجب أن
تتأكد من الاتساق في استعمال المصطلحات، واتساق أسلوب الترجمة، وأن تتناغم
مع مصطلحات وأسلوب التطبيقات الأخرى في المنصة التي سيعمل عليها التطبيق
(وِندوز أو كدي أو گنوم، إلخ.) و خذ في الحسبان الاتساق مع اللغة التي
يستعملها الناس خارج عالم الحوسبة؛ في الجامعات والصحف والمجلات و غيرها،
عندما يكون ذلك مفيدا. مع الوقت ستطور أسلوبك الخاص ولن تقع في أخطاء
الآخرين، لكن لا تسعى للاختلاف لمجرد الاختلاف.

{\bi كن حريصا وجريئا.} قد تبدو هذه النصيحة متناقضة، لكن اسعَ إلى
الموازنة بين هذين الهدفين المتضادين. ينبغي أن تكون حريصا في عملك وأن
تسعى للأفضل. لا تتردد في البحث والدراسة، وتأكد أن تراجع أنت و شخص آخر
الترجمة. أحيانا يكون من الأفضل تأجيل الترجمة حتى تصل إلى فهم أفضل. لكن
في أحيان أخرى يكون عليك المضي قدما وصَوغُ مصطلح جديد أو تغيير عادة سيئة
شاعت بين المترجمين. من السهل عادة إصلاح الأخطاء في البرمجيات الحرة،
وإذا كنت ممن يتعلمون من أخطائهم فسيفيدك كثيرا عدم المبالغة في الحرص.

{\bi توقع النقد.} للأسف قد لا يقدر الجميع المجهود الذي تبذله. عندما تأخذ
زمام المبادرة فإنك تُعرِّضُ نفسك للنقد، وهذا أمر متوقع وليس سيئا
بالضرورة. فهذا يعني أن شخصا ما اهتم بما تفعله، والنقد الجيد يساعدك على
تحسين عملك، بتنقيح المصطلحات التي تستعملها أو تقليل أخطائك. ضع في
الاعتبار كذلك أن المستخدمين يحتاجون أحيانا لبعض الوقت للاعتياد على
المصطلحات الجديدة في الواجهة المعربة، وخاصة إن كان مجالا جديدا على
لغتك. يتطور المستخدمون معك، والمصطلحات التي انتقدوها في البداية ستُصبِح
مقبولة ومعتادة إن كانت مبنية على تفكير سليم. لا تتجاهل النقد، بل تجاوب
معه، و خذ في حسبانك أن الواجهات المترجمة قد تفاجئ الناس وتبعث على
النقد.

{\bi استمتع.} لا تتورط في عمل لا تحبه. غالبا ما يعمل على توطين البرمجيات
الحرة متطوعون يستمتعون بهذا. إذا لم يكن الأمر مرضيا بشكل أو بآخر فقد لا
تستطيع الاستمرار فيه ومن ثم لن تحقق أهدافك، لذا ترجم البرمجيات التي
تستخدمها وتستمتع بها، و أطرِ على المبرمجين. إذا أردت أن يشيد الناس
بعملك فعليك أن تعمل بجد على نشره وتسويقه.

\section{البيئة المحيطة}
العمل الذي تقوم به يحدث في بيئة يتفاعل معها، وليست كل النصائح مناسبة لكل
المواقف، لذا عليك أن تعي ما يتوفر لك من موارد وطبيعة مستخدميك والبيئة
التي يعيشون فيها لتنجز عملا أحسن وبفاعلية أفضل.

\subsection{الجمهور المستهدف: متحدثو لغتك}
في أي ترجمة يجب أن تعرف أولا لمن تترجم؛ من سيقرأ النص الذي تترجمه. من أي
فئة عمرية. وأي مستوى تعليمي. فقد تختلف الترجمة إن كنت تترجم للأطفال في
المدارس الريفية عما إذا كنت تترجم لأساتذة الجامعات. ولا تنس أن
المتحدثين باللغة التي تترجم إليها هم الجمهور الأساسي الذي تسوق لهم
عملك، فلا تتواصل بلغة أخرى كالإنگليزية دون داعٍ عند الرد على أسئلة
المستخدمين أو مناقشة المصطلحات، بل إن التواصل مع الجمهور بلغتهم هو
وسيلة لإتقان الترجمة ذاتها و لتسويقها بينهم.

\subsection{الجمهور المستهدف: مستخدمو البرمجيات الحرة}
عادة ما يكون لمستخدمي البرمجيات الحرة حِسٌّ مجتمعي، كما أنهم يواظبون على
الاطلاع على آخر أخبار البرمجيات. كما يمكن أن تجد بينهم متطوعين للتباحث
معهم في صوغ المصطلحات أو لمراجعة الترجمة أو لاختبار البرمجيات المترجمة،
وقد تجد من يهتم بالمساعدة في الدعاية. قد يتحمس الناس للتطوع بوقتهم لأنك
توطن برمجيات تحترم حرية المستخدم.

ضع في اعتبارك أن الشكوى سهلة لكن قليلون من يساعدون بمشاركات مؤثرة، واعلم
أن المستخدمين قد يُعلِّقون عليك آمالا كبيرة، ثم لا يكون الناتج على قدر
ما تمنوا.

\subsection{فريق التوطين}
إن كانت أهدافك كبيرة فستحتاج الكثير من المساعدة لتحقيقها، وبناء فريق قوي
يساعدك على إنجاز عمل أكبر، وانضمام ذوي الخبرات المختلفة يساعد على
إتقانه. إذا كنت تُتِّم عمل مترجمين سابقين فحاول التواصل معهم. عليك أيضا
توثيق ما تفعله وإطلاع الآخرين عليه. ابحث عن طرق للترويج للعمل الذي تقوم
به بين من تسعى إلى مساعدتهم، بتوظيف الصحف والإذاعة والتلفاز والمدونات
والشبكات الاجتماعية.

حاول تدريب المستخدمين الجدد ليصبحوا مشاركين فعالين، وانشر الوعي
بالتوطين. في مشروع Translate.org.za كثيرا ما ننظم ماراثونات يجتمع فيها
أشخاص ليوم أو يومين ليترجموا إحدى البرمجيات، وهي طريقة فعالة لتعريف
الناس بماهية توطين البرمجيات وتدريبهم على أساسياته وفي نفس الوقت تستمتع
مجموعة من الناس بالعمل لتحقيق هدف مشترك.

تعرف أكثر على هذه الماراثونات:
\CURL{http://translate.sourceforge.net/wiki/guide/translateathon}

إذا كنت تعمل وحدك فلا تنزعج من تكرار الإشارة إلى فرق التوطين، فكثير من
مشروعات توطين البرمجيات الحرة (خاصة إلى اللغات قليلة الانتشار) يقوم
عليها متطوعون يعملون منفردين. إذا كان توطين البرمجيات أمرا جديدا في
لغتك فعلى شخص ما أن يكون الرائد، وقد تكون أنت. بعض الناس يسهل عليهم
الانضمام إلى فريق موجود على بدء فريق من الصفر.

\subsection{دور اللغة الإنگليزية}
من المهم الحديث عن دور اللغة الإنگليزية في معرض مناقشة بيئة توطين
البرمجيات، فقد أصبحت الإنگليزية اللغة السائدة في عالم البرمجيات وهذا
يؤثر على طريقة عملنا. أغلب البرمجيات تترجم من الإنگليزية، كما قد تجد
بعض المستخدمين وحتى بعض رفقاك في الترجمة معتادين على المصطلحات والأسلوب
الإنگليزي ويميلون إليه، بوعي أو بغير.

ستجد أن أغلب التواصل في مشروعات البرمجيات الحرة يكون بالإنگليزية، لذا
سيسهل عليك المشاركة في هذه المشروعات إن كنت تجيد الإنگليزية، كما سيسهل
عليك فهم النصوص الإنگليزية أفضل، إن كانت مكتوبة بلغة سليمة، و اعلم أن
الإنگليزية ليست اللغة الأم لكل المشاركين في المشروعات، فعندما تتواصل مع
الآخرين بالإنگليزية حاول أن تشرح ما تريده بوضوح وأن تراجع النص قبل
إرساله حتى لا يساء فهمه نتيجة أخطاء الإملاء وغيرها، و لكي توفّر وقتك و
قت الآخرين.

ومع هذا توجد وسائل عديدة لمساعدة من لا يجيد الإنگليزية بقدر كاف، فبعض
البرمجيات (مثل بُوتِل) يمكن أن تعرض النص بلغة أخرى إضافية للمساعدة على
فهم أفضل للنص، و هذا قد يساعد حتى من يتقنون الإنگليزية على استجلاء
المعنى. كما قد تساعد خدمات الترجمة الآلية في التواصل مع الآخرين بلغات
لا تجيدها. ويمكن كذلك الاستفادة من الأشخاص الذين لا يعرفون أي لغات
أجنبية في أمور مثل الترويج للبرمجيات المترجمة و هم مفيدون للغاية في
اختبار الترجمات.

\subsection{تَعرَّف على ما تترجمه}
عندما تكون مستعدا لترجمة برمجية ما ابدأ باستخدامها والاعتياد عليها. يسهل
في عالم البرمجيات الحرة الحصول على البرمجيات وتجربتها. عليك كذلك زيارة
موقع مشروع البرمجية و مطالعة وثائقها. سيجعلك هذا تعرف ما الذي تفعله
البرمجية وبالتالي معنى ما تترجمه. كما تشكل ملفات المساعدة التي قد توجد
مع البرمجية مصدرا مهما آخر للمعلومات.

وإذا كنت تستحدث ترجمة جديدة من الصفر فستفيدك أي إصدارة قديمة من البرمجية
في التعرف عليها وطريقة عملها وما تعنيه المصطلحات المختلفة. حاول دوما أن
تبني الترجمة على معرفة جيدة بالبرمجية ولا تعتمد على النصوص وحدها.

\chapter{قبل البدء}
\section[ref:20165030]{الأدوات التي سنستخدمها}
الأدوات التي يمكن استخدامها
للترجمة عديدة ولا نستطيع التطرق إليها كلها. لذا سيكون أغلب الشرح عن
بُوتِل وفِرتال، فكلتاهما طُوِّرَتا لزيادة إنتاجية المترجمين ومساعدتهم
على إدارة فِرَقهم وتحسين الجودة.

{\bf بُوتِل} نظام لإنجاز الترجمة وإدارتها عبر الوِب، تستخدمه كثير من
المشروعات الحرة لإدارة صيرورة التوطين، و به وظائف عِدة تساعد في مراجعة
الترجمة وتنظيم عمل الفريق. وهو مناسب للعمل الجماعي، وللمساهمين غير
المتمرسين، ولماراثونات الترجمة. راجع صفحة مشروع بوتل لمطالعة المزيد
عنه:
\CURL{http://pootle.locamotion.org}

مع أن الترجمة عبر الوِب مفيدة إلا أن توفر اتصال جيد بالإنترنت مايزال
عائقا أمام كثيرين. {\bf فِرْتَال} أداة ترجمة تعمل على الحاسوب بغير حاجة
لاتصال بالإنترنت، و به وظائف عديدة لزيادة الإنتاجية مع المحافظة على
الجودة. تجود تعليمات التنصيب و مزيد من المعلومات على موقع فرتال:
\CURL{http://virtaal.org}

لست مضطرا إلى الاختيار ما بين هاتين الأداتين فهما تكمل إحداهما الأخرى.
فالمشروعات المستضافة على بوتل تمكن ترجمتها بلا اتصال بالإنترنت باستخدام
فرتال ثم رفع الملفات المترجمة إلى بوتل. قسم “أدوات التوطين” في صفحة
\at[ref:30364807] يشرح هاتين الأداتين بتفصيل أكبر. عدة أقسام
في هذا الكتاب تتحدث عن وظائف معينة في هاتين الأداتين لذا من المحبذ
تنصيب فرتال و فتح حساب في أحد خواديم بوتل لتجربة الإرشادات الواردة في
الكتاب. سيقدم فرتال العون الأكبر أثناء الترجمة عبر تلوين الأجزاء المهمة
لإبرازها، والتكامل مع كثير من الموارد المساعدة إذا كنت متصلا بالإنترنت.

بُنِيَت كلتا الأداتين على {\bf عُدّة الترجمة} و هي تجميعة قوية من أدوات
صغيرة لمساعدة المُوَطِّنين. تشتغل أدوات عدة الترجمة من سطر الأوامر لذا
لن نتطرق إليها كثيرا في هذا الكتاب. تتوفر تعليمات التنصيب و مزيد من
المعلومات على موقع عدة الترجمة:
\CURL{http://translate.sourceforge.net/wiki/toolkit/index}

\section{عرض لغتك}
قبل البدء في العمل تأكد من أنك تستطيع عرض لغتك بشكل سليم على شاشة
الحاسوب. لم تعد هذه مشكلة لكثير من اللغات وكل ما تحتاجه هو التحقق من
تنصيب الخطوط المطلوبة على الحاسوب. يمكنك مطالعة صفحة ويكيبيديا بلغتك أو
موقع إحدى الصحف المحلية على الإنترنت للتأكد من أن كل الحروف تظهر كما
ينبغي، وإذا وجدت مشاكل فقد تجد في تلك المواقع روابط إلى الخطوط المطلوبة
وتعليمات لطريقة عرض لغتك على الشاشة. من المهم الانتباه إلى أنك تحتاج
إلى ما تُعرف بخطوط يُونِيكود، ويمكنك مراجعة دليل خطوط يونيكود:
\URL{http://www.unifont.org/fontguide}

ولأنك غالبا ستقضي معظم الوقت مع أداة الترجمة فعليك التأكد من أن النص
يظهر سليما فيها وليس في متصفح الوِب وحده، وذلك بنسخ نَصٍّ ولصقه في حقل
في الأداة. بعض التطبيقات (مثل فرتال) تتيح لك تطويع تضبيطات الخط لتحصل
على أفضل عرض للنصوص يناسبك.

وأيضا إذا لم يأت نظام التشغيل مُزوَّدا بالخطوط التي تتطلبها لغتك فغالبا
ستكون الحال هي نفسها مع مستخدمي البرمجيات المترجمة، لذا عليك التأكد من
وجود تعليمات واضحة منشورة متوافقة مع مختلف أنظمة التشغيل لتنصيب الخطوط
الإضافية. ضع في الاعتبار أن المستخدمين لن يستطيعوا قراءة التعليمات
ذاتها المكتوبة بلغتهم ما لم تكن لديهم الخطوط المطلوبة، إلا إذا حفظت
التعليمات في ملف PDF مُضمَّنة فيه الخطوط المناسبة (و عندها لن يحتاج
المستخدمين لوجود الخطوط عند قراءة الملف). بعض تقنيات الوِب الحديثة تتيح
للمواقع توفير الخطوط المطلوبة لعرض النص في المتصفح دونما حاجة لأن
ينصِّبها المستخدم بنفسه، لكن لا يمكن الارتكان إلى هذا حاليا.

في بعض الحالات يمكن توزيع الخطوط مع البرمجيات لضمان عرض اللغة كما ينبغي،
تَباحث في هذا مع المطورين وسيُعلمونك بإمكانية ذلك أو عدمه.

\section{كتابة لغتك}
والآن بعد أن أصبحت قادرا على عرض لغتك على الحاسوب كما ينبغي، عليك التحقق
من إمكانية الكتابة باللغة على الحاسوب أيضا. في الأغلب ستستطيع اختيار
تخطيط لوحة المفاتيح أو طريقة الإدخال المطلوبة من تَحَكُّمات إعدادات
نظام التشغيل. تأكد من أنك تستطيع إدخال كل الأحرف التي تتطلبها كتابة
لغتك.

إذا لم تكن معتادا على الكتابة بلغتك على لوحة المفاتيح فقد يفيد التمرن
على ذلك لزيادة سرعتك، إذ كلما كتبت أسرع كلما أنجزت عملا أكثر واستمتعت
بالترجمة أكثر.

\section[ref:34561726]{الرموز والمَحَلِّيات – الدعم الأوَّلِيُّ للغتك}
عندما تساهم بأول ترجمة أو دعم
للغتك في برمجية ما يتوجب تعريف اللغة للمطورين بدقة. تذكَّر أن تستعمل
الاسم الإنگليزي للغة وضع في الاعتبار أن المطورين قد لا يعرفون كثيرا عن
لغتك. يستعمل في العادة رمز من حرفين أو ثلاثة للإشارة إلى كل لغة، فمثلا
رمز الإنگليزية من حرفين هو ‘en’ والعربية ‘ar’، وعادة ما تَستخدِم
المشروعات الحرة الرمز ذي الحرفين إن وجد وإلا فالرمز ذي الأحرف الثلاثة.
يحدد هذه الرموز معيار يسمى أيزو 639، تمكن مطالعة المزيد عنه في
ويكيبيديا:
\CURL{http://en.wikipedia.org/wiki/ISO_639}

بعض اللغات التي تُستعمّلُ تنويعات منها في أكثر من بلد تكون لها ترجمة
مختلفة لكل بلد، وفي هذه الحالة تُمَيَّزُ يُلحق رمز البلد إلى رمز اللغة،
مثل ‘fr\_CA’ للفرنسية الكندية أو ‘en\_GB’ للإنگليزية البريطانية. لاحظ
أنه لا حاجة لتحديد رمز البلد إلا في حال وجود أكثر من ترجمة لأكثر من بلد
تستعمل فيه ذات اللغة، لذا لا تُحدد رمز البلد إلا إذا كان مشروع البرمجية
المترجمة يتطلب هذا أو كنت تعرف أن لغتك توجد لها ترجمات مختلفة باختلاف
البلد.

بعد التطبيقات تتطلبُ ما يُعرف باسم {\it المحلية} لتعمل كما ينبغي مع
لغتك. المحلية هي مجموعة إعدادات في نظام التشغيل تُقدِّم إلى البرمجيات
معلومات عن لغتك والأنساق المتعلقة بها مثل طريقة تنسيق التاريخ والوقت
وطريقة كتابة الأرقام إلى غير ذلك. المحليات المتاحة في نظام التشغيل عادة
تُحدِّد اللغات التي يدعمها نظام التشغيل. بعض التطبيقات تحتوي على محليات
خاصة بها و لا تعتمد على نظام التشغيل في هذا.

فيما يخص لينُكس وبعض البرمجيات الأخرى التي تستعمل المحليات، يُمكن إنشاء
المحلية إذا لم تكن موجودة. في حالة لينكس فالمشروع البرمجي الذي تُضاف
إليه المحلية يسمى “glibc”، ويمكنك الاطلاع على مزيد من المعلومات عنه
الصفحة:
\CURL{http://translate.sourceforge.net/wiki/guide/locales/glibc}

بالرغم من أن بعض البرمجيات في لينُكس لا تتطلب محلية glibc إلا أنه من
الأفضل على كل حال إنشاء محلية إذا لم تكن موجودة للغتك. توجد المئات من
المحليات حاليا، لذا فأغلب الناس لن يكون عليهم القلق بهذا الشأن.

\section{إضافات مفيدة}
بعد ضبط الأمور السابقة تستطيع توطين أي برمجية تقريبا، لكن توجد أدوات
أخرى تساعد في تحسين الجودة أو زيادة الإنتاجية. انظر إذا كنت تستطيع
الحصول على أي من هذه المُعينات على الترجمة:

\startitemize[1]
\item مُدَقِّقٌ إملائيٌ – تحقق من وجود مدقق إملائي و تفعيله في متصفح
الوِب (إذا كنت تترجم عبر الوب) أو في فِرتال.
\item قواعد الإملاء و الرسم والنحو في لغتك
\item معاجم ثنائية اللغة متخصصة في المجال الذي تترجم فيه
\item معجم إنگليزي
\stopitemize
وحتى إن لم تكن هذه متاحة سوى على الوب فقد تساعد عند الاضطرار.

\chapter{الأسلوب}
لكل قطعة أسلوب يغلب عليها، وهذا ينطبق على اللغة المنشورة بها البرمجيات
ابتداء. فقد تتميز بكونها رسمية أو غير رسمية، تقنية أو مُبَسَّطة، و
بدرجة ما من الالتزام بقواعد اللغة.

و عندما نترجم فإننا ننشئ قطعة لها أسلوبها، لذا علينا تحديد هدفنا منذ
البداية؛ هل نريد الترجمة رسمية بقدر النص الأصلي؟ هل يمكن استعمال كلمات
مستعارة من لُغات أخرى أم يجب الاقتصار على الكلمات الشائعة في لغتنا؟
وغيرها من الأسئلة التي تشكل منهجنا في الترجمة. علينا أن نضع في الاعتبار
الهدف من الترجمة والجمهور الذي نستهدفه.

من المستحسن وضع إطار لتوطين البرمجيات بلغتك، فعندما تتعدد أساليب أداء
الأعمال قد يتَّبع كل مترجم أسلوبا مختلفة لأداء العمل ذاته، وعدم الاتساق
الناشئ عن هذا يقلل من جودة الترجمة. وبدون معايير محددة قد تواجه
المترجمين الجدد في فريقك اختيارات صعبة تُنفِّرهم وتقلل إنتاجيتهم. يعرض
هذا القسم لمسألة الأسلوب. تُستحسن مناقشة هذه المسائل مع باقي المترجمين
والمهتمين بالأمر وتوثيق هذه الأطر و صقلها مع الوقت.

لاحظ أن المستخدمين سيعتادون على الأسلوب المستعمل مادام متسقا مع بعضه ومع
خلفيتهم الثقافية.

\section{قواعد الكتابة}
لأغلب اللغات قواعد قياسية للكتابة، تشمل الأبجدية والإملاء وعلامات
الترقيم وغيرها تسمى {\it نظام الكتابة}، وقد تُرسَم بعض اللغات بأكثر من
نظام (مثل الهَوسَ التي ترسم بالحرف العربي أو اللاتيني) أو قد تتغير
قواعد نظام الكتابة بتغير الزمن لتغير قواعد الإملاء أو غيرها.

إذا كان للغتك أكثر من نظام للكتابة فعليك تحديد أيها ستستعمل، وتمكن
الاستفادة من منشورات المجامع اللغوية وغيرها من المؤسسات المعنية بتقنين
قواعد اللغة، أو حتى الكتب المدرسية الجيدة في حال لم تكن معرفتك بنظام
الكتابة كافية. تساعد المعاجم أيضا في ضبط الإملاء. إن لم تكن للغتك قواعد
مقننة فسيترك لك هذا حرية اختيار القواعد التي تلائمك، و عندها سيصب عملك
في جهود تقنين كتابة هذه اللغة.

أحيانا ما يغفل المترجمون الجدد أثناء عملهم عن القواعد التي تتبعها لغتك،
لذا من المهم توثيق الأخطاء الشائعة ليُنتبه إليها.

في كتابة العربية تشيع أخطاء الإملاء في الهمزات، و التمييز ما بين التاء
المربوطة و الهاء في أواخر الكلمات، و كذلك التمييز ما بين الألف المقصورة
و ياءُ آخرِ الكلمة. كما تشيع إقليميا أخطاء مرجعها اللهجات المحلية، مثل
الخلط ما بين الذال و الزّاي، و الظاء و الضاد، و غيرها. تلك الأخطاء
كلّها قد تُغيّر المعنى (لأنها كلمات أخرى) أو على الأقل تُصعِّب الفهم،
كما تُقلل مقروئية النصّ كثيرا، يزيد من هذا إيجاز نصوص البرمجيات. فَراعِ
قواعد الكتابة و تحقّق مما تشكُّ في درايتك به. كذلك استعمل علامات الشّكل
(الحركات) كلّما وَجَدتَ هذا يُقلل اللبس أو يُسهّل القراءة، فهي جزء
أساسي من نظام كتابة العربية، لا كماليات.

من الأخطاء الشائعة في اللغات المكتوبة بالأحرف اللاتينية التي تُميّز ما
بين أشكال كبيرة و صغيرة للأحرف تجاهلُ نمط الأحرف كليا باستعمال إما
الأحرف الكبيرة على الدوام أو إغفالها كليا وهو ما يعطي انطباعا سيئا عن
جودة النص ويقلل مقروئيته، أو بنقل نمط حالة الأحرف من الإنگليزية.
التمايُز المنضبط في حالة الأحرف يساعد على فهم المعنى؛ فأسماء الأعلام
(أسماء الأشخاص أو العلامات التجارية أو البلاد) تُبدأ بحرف كبير في بعض
اللغات، والعبارة التي لا تبدأ بحرف كبير قد تُفهم على أنها جزء من جملة.
لذا يجب اتباع القواعد الصحيحة للكتابة. 

\subsubsection{أمثلة}
الأمثلة التالية بلغة سوتو الشمالية التي ليست في كتابتها أحرف كبيرة سوى
في أوائل الجمل وأسماء الأعلام، لذا فالحالة تطابق الإنگليزية في أوّل كل
رسالة، إلا أنها تخالفها في باقي العبارة لمخالفتها قواعد كتابة سوتو
الشمالية.
\SetTableToWidth{\textwidth}
\starttable[|r|r|r|]
\HL
\NC\ReFormat[cB]{الإنگليزية}\NC\ReFormat[cB]{خطأ}\NC\ReFormat[cB]{صواب}     \SR
\HL
\NC Help              \NC thušo                  \NC Thušo                  \FR
\NC \ltr{Save As…}
\NC \ltr{Boloka E le...}
\NC \ltr{Boloka e le...}\MR
\NC Font Size         \NC Bogolo Bja Fonte       \NC Bogolo bja fonte       \MR
\NC Online Help       \NC Thušo ya Inthaneteng   \NC Thušo ya inthaneteng   \MR
\NC Select All        \NC Kgetha Tšohle          \NC Kgetha tšohle          \MR
\NC Print PDF File    \NC gatiša faele ya pdf    \NC Gatiša faele ya PDF    \MR
\NC Print Acrobat File\NC Gatiša Faele ya Acrobat\NC Gatiša faele ya Acrobat\MR
\NC no selection      \NC Ga go kgetho           \NC ga go kgetho           \LR
\HL
\stoptable

علامات الترقيم والمسافات من الأمور المهمة التي ينبغي توثيقها في قواعد
الترجمة. وحتى لو كانت قواعد كتابة لغتك لا تختلف عن قواعد كتابة
الإنگليزية فمن المهم أن ينتبه المترجمون إلى الجُمَل التي تنتهي بنقطتين
(:) أو علامة استفهام (?) أو علامة تعجب (!) أو ثلاث نقاط (…). لاحظ أن
النقاط الثلاثة قد تكتب على الحاسوب ثلاثة نقاط متتالية أو محرفا واحدا
يصورها كلها، وقد يصعب إدخاله في لوحة المفاتيح. يُسهِّل فِِرتال إدراج
المحارف الخاصة كهذه بتمييزها على أنها مُدرج (المُدرجات مفهوم خاص في
فرتال نتناوله بتفصيل في صفحة \at[ref:32596109]).

أما إذا كانت للغتك علامات ترقيم تختلف عمّا للإنگليزية فمن المهم تذكير
المترجمين بالاختلافات وضرب أمثلة، مع توضيع علامات الترقيم المتطابقة
لتلافي اللبس. ومع أن علامات الترقيم في النص المصدرغالبا ما تدل على أمر
هام، إلا أنك يجب ألا تتبعها دون تمييز، فمثلا يكثر استعمال علامات التعجب
(!) في الإنگليزية، لكن إذا لم تكن شائعة في لغة ما بنفس القدر فيمكنك
تجاهلها.

الاستعمال غير المعتاد للمسافات في النص الأصلي قد يكون عند قصد، مثل وضع
مسافات في بداية الجملة أو في نهايتها، فقد يقصد به محاذاة النص أو الفصل
بين تَحَكُّمات في واجهة البرمجية أو لضبط المسافات في جملة تنقسم إلى عدة
أجزاء. و قد يكون خطأ برمجيا يتوجب الإبلاغ عنه. أحيانا تُوضع مسافتان بين
الجملة و التي تليها، لكن لا يلزم اتباع هذه القاعدة، وإن كان اتباعها قد
يساعد على ضمان الاتساق مع ما سبق. فحوص تدقيق الجودة الآلية في بعض
البرمجيات (مثل بوتل) تحاول التنبيه إلى أي اختلافات من هذا النوع بين
الأصل والترجمة.

إن كانت لغتك تكتب بغير الأحرف اللاتينية فمن المهم وضع معايير لكتابة
الأعلام الأجنبية؛ متى تنقل صوتيا ومتى تكتب بأحرف لغتها الأصلية،
والاتفاق على قواعد التصويت. نتناول هذا الأمر بتفصيل أكبر في صفحة
\at[ref:20221205].

\section{الأساليب البلاغية}
توجد دوما أكثر من طريقة لصوغ نفس المعنى، فيمكن صوغ الجملة فتكون رسمية
متكلفة أو غير رسمية، أو مُلحّة، أو مقتضبة أو مسهبة، أو حتى فظّة.
الصياغة الأمثل التي تناسب ترجمة البرمجية تعتمد على عدة عوامل. اسأل نفسك
الأسئلة التالية:

\startitemize[1]
\item ما عمر مستخدمي البرمجية؟
\item ما البيئة التي تستخدم فيها؛ في العمل أم في المنزل؟
\item هل بعض الصياغات مسيئة أو غير مهذبة؟
\stopitemize
فإلى جانب أننا نريد صياغة سليمة قواعديا، علينا مراعاة الصياغة بما لا
يسيء المستخدم و بما يناسب التطبيق. فلغة الأعمال لا تناسب برمجية
للأطفال، واستعمل لغة غير رسمية قد لا يناسب برمجية للمحاسبة، وسؤال
المستخدم عن معلومات معينة بصورة غير مهذبة قد يعطي المستخدم انطباعا سيئا
عن البرمجية ويقلل تفاعله معه. 

بعض ما قد يفيد توثيقه في معايير الترجمة:

\startitemize[1]
\item ما مدى رسمية النص؟ هل يفترض استعمال ألقاب التوقير عند مخاطبة
المستخدم؟
\item كيف تترجم الأسئلة؟ في اللغة الإنگليزية يكثر استعمال “please”، فما
الكلمة المناسبة في لغتك عندما تطلب شيئا؟ الترجمة الحرفية “رجاء” أو
الإكثار منها قد تبدو تحذلقا أو مبالغة في التزلّف فتظهر البرمجية كأنما
تستجدي معروفا من المستخدم، بينما المقصود تشجيع المستخدم بتهذيبٍ على فعل
شيء ما. أحيانا تطلب البرمجية من المستخدم توكيد شيء ما، لكن ترجمة
العبارة “\ltr{Are you sure?}” بأسلوب سيئة قد تجعل البرمجية تبدو كأنما تشكك في
قدرة المستخدم على الفهم.
\item كيف تترجم الأوامر؟ في الإنگليزية غالبا ما تكون النصوص على الأزرار
وفي القوائم أفعال أمر، لكن في لغات أخرى لا تُصاغ على أنها أوامر وتستعمل
صيغة المصدر من الأفعال. عليك تقرير الأنسب للغتك، وعليك التفكير كذلك في
اللبس المحتمل بين كون الأمر موجها للمستخدم أم للحاسوب، و إن كان سيفهم
من الصياغة أن التفاعل مع عناصر الواجهة سيؤدي إلى إحداث فعل ما أم غير
ذلك.
\stopitemize

\section{احفظ روح النص المَصدَر}
إن كان أسلوب الكتابة في لغتك أميل إلى الرسمي فربما كان من الأفضل التمييز
بين درجات مختلفة من الرسمية بمطالعة النص المصدر، فبرمجيات الشبكات
الاجتماعية والتراسل الفوري مثلا لا تناسبها الرسمية المفرطة.

من المهم للغاية في الترجمة السعي إلى إيصال المعنى المراد في النص المصدر.
فأحيانا تظن أن بوسعك صوغ الترجمة بأحسن مما هي عليه في المصدر، لكن عليك
قبل فعل هذا التيقن من أنك لست تغير المعنى المراد. مع هذا فالصياغة قد
تكون غامضة في المصدر أو أن بنية العبارات و إيجازها يحتمل التأويل. راجع
النقاط في قسم “مبادئ أساسية” في صفحة \at[ref:34314226] ففيها
أمثلة على هذا. على سبيل المثال يجب ألا تكون الترجمة أكثر عمومية و لا
أكثر تخصيصا من النص المصدر. واعلم أن النصّ قد يُستعمل في سياقات قد لا
تتوقعها، فإذا بعدت الترجمة كثيرا عن الأصل فإنها قد لا تلائم بعض تلك
السياقات.

\section{ابتعد عن الترجمة الحرفية}
شجعناك سابقا على أن تكون {\bf حريصا} و {\bf جريئا}، وقد يبدو هذا القسم
أيضا متناقضا مع السابق إذ يشجعك على الالتزام بمتطلبات اللغة الوجهة عوضا
عن الالتزام بالأصل. لكن في الحقيقة كلا الهدفين مهمين ولا يتعارضان في
الأغلب؛ فالمطلوب أن تُخرج الترجمة نصا سليما يتبع الأساليب اللغوية للغة
المترجم إليها وفي الوقت ذاته يوصل المعنى المقصود في النص المَصدَر.

يجب أن تلتزم الترجمة بالأساليب والتراكيب اللغوية الصحيحة في لغتك؛ فلا
تقع في خطأ اتباع نفس ترتيب الكلمات في الجملة الإنگليزية إذا لم يكن
الأنسب في لغتك، ولا تترجم التعبيرات حرفيا فبعض التعبيرات نتاج لثقافة
معينة ليست موجودة في اللغة المنقول إليها. واحذر من الكلمات غير اللائقة
في لغتك أو التي تحمل معاني غير مقصودة تبعد المستخدم عن الرسالة المطلوب
إيصالها.

في بعض المشروعات تُنفِّرُ وثائق المطورين من صيغة البناء للمجهول، لكن قد
يكون النص الإنگليزي فقط هو المعني بهذا، لذا لا تترد في تجاهل مثل هذا
التوجيه إذا كان المبني للمجهول هو الأنسب في لغتك، إلا إذا كانت تلك
التوجيهات مقصودة بها الترجمة أيضا.

\chapter{المصطلحات}
المُصطَلَحُ كلمة أو عبارة قصيرة لها معنى محدد مُتَّفقٌ عليه، والترجمة
الجيدة تتطلب العناية بالمصطلحات وترجماتها. ويمكن إدراج العبارات القصيرة
مع المصطلحات إذ كثيرا ما تكون لها معان محددة مثل المصطلحات ويمكن
إضافتها للمعاجم ومعاملتها في أدوات الترجمة على النحو ذاته كالمصطلحات.
كما يمكن استعمالها لتوضيح بعض جوانب مسألة الأسلوب.

\subsubsection{أمثلة}
\startitemize[1]
\item file
\item download
\item network connection
\item secure
\item Are you sure you want
\stopitemize
كما ترى قد يتألف المصطلح من كلمة أو أكثر، كما يمكن أن يكون اسما أو فعلا
أو حتى عبارة قصيرة تتكرر في الترجمة.

\section{ما أهمية تطوير المصطلحات؟}
تطوير المصطلحات من المهام الأساسية المستمرة أثناء الترجمة، فحسن اختيار
المصطلحات والاتساق في استعمالها يساعد المستخدمين على التكيف مع
البرمجيات المترجمة والفخر باستخدامها بدلا من الخجل منها ومن سوء
مستواها. كما تحط الترجمة السيئة من قدر البرمجية في نظر المستخدم.

تظهر دوما مصطلحات جديدة البرمجيات تتطلب صوغ مقابلات لها، لهذا فمهمة وضع
المصطلحات مستمرة باستمرار ترجمة البرمجيات. مع هذا فمُراكمة مسردٍ
لترجمات المصطلحات الشائعة يساعد كل أعضاء الفريق على الترجمة أسرع،
وسيساعد على اتساق الترجمات مع بعضها. و لاحظ أنك لست بحاجة لمعجم يحوي
آلاف المصطلحات قبل الشروع في الترجمة.

تُثَبِّطُ الترجمات التي تحوي كثيرا من المصطلحات الصعبة المترجمينَ الجدد
وقد يقعون في أخطاء كثيرة إذا لم تكن لهم خبرة بمصطلحات الحاسوب، لذا من
المهم للغاية وجود مسرد للمصطلحات لمساعدة أعضاء الفريق وخاصة الجدد منهم
وتحفيزهم على المواصلة. كما يساعد المسرد الجيد كثيرا في جودة الترجمة
وتحفيز المترجمين في ماراثونات الترجمة.

أحد فوائد مراكمة مسرد مصطلحات جيد أنه يضمن دقة ترجمة المصطلحات، فبعض
المصطلحات صعبة الفهم وأحيانا تكون لها معان خاصة في سياق البرمجية موضوع
الترجمة، وإذا أخطأ المترجم المعنى المقصود تدَنَّت جودة الترجمة. ولهذا
ستستفيد كثيرا من إفراد مساحة مستقلة في مشروعك للعمل على مسرد المصطلحات.
المصطلحات الجيدة تضمن أن تحمل الترجمة المعنى الصحيح وألا تُربك
المستخدم. تذكَّر أنك لا تريد أي ترجمة و كفى، وفي حال كان للمصطلح أكثر
من مقابل حسب السياق فمن المهم أن يشتمل مسرد المصطلحات هذه المعلومة.

قد تكون العبارات القصيرة فرصة لتوحيد بعض الجوانب الأسلوبية في الترجمة،
مثل كيفية صوغ الطلبات الموجهة إلى المستخدِم، (كما في الأمثلة أعلاه “Are
you sure you want”).

\subsubsection{مثال}
\example{Are you sure you want to download the file over the secure network
connection?}

في هذا المثال عدة مواضع تحتمل الخطأ، كألّا تكون لكلمة “secure” ترجمة
متفق عليها، أو لا تكون ترجمة “network connection” بديهية كفاية. تبدأ
الجملة كذلك بعبارة “Are you sure you want to ...” والتي تمكن ترجمتها
بأكثر من أسلوب، لكن عدم الاتفاق على ترجمة موحدة لها قد يعطي انطباعا
سيئا عن البرمجية.

الكلمات الموضحة بسطر تحتها قد تصلح لإضافتها إلى مسرد المصطلحات لتسهيل و
توحيد ترجمتها:

\example{\underbar{Are you sure you want} to \underbar{download} the
\underbar{file} over the \underbar{secure} \underbar{network
connection}?}

يظهر لنا أن وجود مسرد يحوي ترجمة كل من هذه المصطلحات يُيسر ترجمة النصّ
السابق، كما ستأتي الترجمة متسقة مع الترجمات الأخرى ومع أسلوب الترجمة
المتفق عليه، كما ستزيد إنتاجية المترجم إن كان يستخدم أداة ترجمة تسهل
إدراج المصطلحات دون حاجة لكتابتها يدويا. ومع أن هذا المثال مُختلق إلا
أنه يوضح الفوائد التي قد تجنيها حتى من مسرد لا يحوي إلا بضع مئات من
المصطلحات.

\section{أي المصطلحات نطوِّر؟}
قد يبدو أمر المصطلحات هذا مملا فأنت تريد ترجمة البرمجيات وليس تأليف
المعاجم، أليس كذلك؟ (طبعا تأليف معجم أمر مفيد للغاية في حد ذاته).
أوضحنا في القسم السابق أهمية وفائدة مسرد المصطلحات، لكن عليك بالطبع
إبداء أولوية للمصطلحات الشائعة خاصة في بداية عملك.

قد يكون من الأفضل البدء بقائمة صغيرة من المصطلحات، لذا من المهم البدء
بالمصطلحات الأكثر شيوعا. قد تعرف بعض المصطلحات الشائعة التي تستحق وضعها
في القائمة لكن محاولة بناء قائمة المصطلحات بنفسك مضيعة للوقت، من المهم
كذلك اختيار مصطلحات موجودة فعلا في الترجمة التي تعمل عليها بدلا من
العمل على مصطلحات قد تكون مفيدة لكنك لا تحتاجها في الوقت الحالي. من
الحلول المناسبة لهذه المهمة استخدام أداة استخراج مصطلحات مثل {\bf
poterminology} من عُدّة الترجمة (انظر صفحة
\at[ref:20165030])، كما يحتوي بوتل على نفس الوظيفة (انظر صفحة
\at[ref:20355427]). تعمل هذه الأدوات على تحليل النص لاستخراج
قائمة المصطلحات المتكررة منه، قد لا تكون كل المصطلحات في القائمة
الناتجة مهمة أو ذات دلالة تبرر تضمينها في المسرد لكن يمكن عدُّ هذه
القائمة اقتراحات مبدئية توفر الوقت.

يعيب أدوات استخراج المصطلحات أنها في الأغلب لا تعرف أي المصطلحات يكثر
ورودها في أكثر أجزاء البرمجية ظهورا للمستخدم. فقد تحصل -مثلا- على
مصطلحات تشيع في رسائل الأعطال، وهذه بالطبع مفيدة للغاية إن كنت ستترجم
رسائل الأعطال، و تفيد الفريق من ناحية الإنتاجية و التحفيز. عليك فعل ما
يحقق أفضل استغلال للوقت المتاح لك.

\section{المصادر المتاحة}
توجد مصادر تساعد في العمل على المصطلحات. تذكر أنه من الأفضل استعمال
المصطلحات الراسخة مادامت جيدة بدلا من وضع مصطلحات جديدة إلا عند
الضرورة، أو لتصويب اختيارات سابقة خاطئة، أو لفض اشتباكات دلالية. لذا من
المستحسن مراجعة المصادر المتاحة حتى وإن يكن لمجرد استلهام الأفكار.
أحيانا قد يفيد الاطلاع على المصطلحات المستعملة في اللغات القريبة من
لغتك، أو حتى في لغات أخرى غريبة عنها عوضا عن الاكتفاء باللغة المصدر
التي هي غالبا الإنگليزية، لأن هذا يعرِّفك على أنماط تفكير مختلفة و
مقاربات مختلفة لمسألة وضع المصطلحات.

هذه بعض المصادر التي قد تفيدك:

\subsubsection{Open-Tran.eu}
يجمع موقع Open-Tran.eu ترجمات مشروعات البرمجيات الحرة ويتيح البحث فيها.
ورغم أنه لا يمكن اعتباره مرجعا لترجمة المصطلحات إلا أنه يعطي صورة عامة
عن ترجمات المصطلحات التي تستعملها البرمجيات الحرة. إذا لم يحتو
Open-Tran.eu على كثير من الاقتراحات بلغتك فربما لقلّة الترجمات المتوفرة
لها، الترجمات الجديدة من عدة مشروعات تضاف إلى الموقع مع الوقت وترجماتك
أنت تساعد في زيادة فائدة هذه الخدمة.

يعرض Open-Tran.eu إلى جانب كل اقتراح المشروع الذي جاء منه، وهذا يساعد
على توحيد المصطلحات مع المشروعات الشبيهة. فإن كنت تترجم أحد مشروعات
گنوم مثلا فسيفيدك الانتباه إلى الاقتراحات الآتية من مشروعات گنوم
الأخرى.
\CURL{http://open-tran.eu}

\subsubsection{موقع جماعة عُيون العرب (عربايز)}
عيون العرب جماعة من المعنيين بتوطين البرمجيات من أرجاء العالم العربي،
يُنسِّقون فيما بينهم توطينَ مشروعات برمجيات حرة عديدة، كما ينخرط بعضهم
في مشروعات مختلفة معنية بتقنيات اللغة العربية و تشغلهم مسألة الكتابة
التقنية بالعربية، و يتواصلون بقوائم بريدية. في الموقع مُعجم تقني تجري
مُراكمته جماعيا هو مصدر جيّد لترجمات المصطلحات، مفتوح لكل من يهمه الأمر
و يُرَّحب فيه باقتراحات التعريبات و بالتعليقات، كما يضم الموقع مصادر
عديدة و إرشادات و أدلّة أسلوبية و مفاهيمية تهم المُوَطِّنين، و مجتمعا
يُرحّب بالقادمين الجدد.
\CURL{http://wiki.arabeyes.org/القاموس_التقني}

\subsubsection{ويكيبيديا}
ويكيبيديا موسوعة تشاركية على الإنترنت، و هي إن كانت لا تختص بترجمة
المصطلحات إلا أنها تتناول طيفا واسعا من الموضوعات التقنية بلغات عدة و
تربط ما بين تلك الموضوعات عبر اللغات مما يُمكّن من البحث فيها عن مصطلح
لترى كيف تُرجم في ويكيبيديا إلى لغات أخرى، كما قد تفيدُ مقالات
ويكيبيديا في استيعاب المفهوم الذي يدل عليه المصطلح، حتى إن لم تكن تلك
المقالات مترجمة إلى لغتك، وهذا يساعد كثيرا في صوغ ترجمة مناسبة للمصطلح.
\CURL{http://www.wikipedia.org}

\section[ref:34314226]{مبادئ أساسية}
\subsection{الدِّقة تقنيا}
يجب أن يَدُلَّ المصطلحُ على المعنى الصحيح، فلا يحمل دلالة أعَمَّ مما
للمصطلح الأصلي و لا دلالة أخَصَّ تتناول جانبًا من المفهوم و حسب. كذلك
تنبغي مراعاة وجود مصطلحات متقاربة لكنها متباينة يجب تمييزها عن بعضها
وإلا قد ينتهي الأمر إلى ترجمتها جميعا بنفس الكلمة. مثلا مصطلحات مثل
“photo” و “image” و “picture” قد ينبغي التمييز بينها في الترجمة، أو
التفرقة بين “file” و “directory” و “document”.

لكن أحيانا ما ترِدُ في النص الأصلي مصطلحات مختلفة للدلالة على مفهوم
واحد، فإذا كنت متأكدا من عدم اختلاف المعنى فيمكنك استعمال ترجمة واحدة
للدلالة على هذا المفهوم. لكن عليك التأكد أن هذا التوحيد لا يسبب أية لبس
أو غموض.

في أحيان أخرى يكون للمصطلح ذاته في اللغة المصدر دلالات عدة، و عندها عليك
التأكد أن الترجمة تستعمل المصطلح الصحيح في لغتك للدلالة على المفهوم
المقصود، فإذا لم تكن للمصطلح ترجمة واحدة تحمل كل معانيه فعليك وضع ترجمة
مختلفة لكل معنى واستعمال الترجمة الصحيحة حسب السياق. بعض الكلمات
الإنگليزية يمكن أن تكون اسما أو فعلا، مثل كلمة “list” قد تكون اسما
بمعنى “قائمة” أو فعلا “اسرد”، وعليك الانتباه إلى المعنى المقصود.

عليك دوما التأكد من فهم المعنى العام للنص الأصلي، باللجوء إلى المعاجم
ومطالعة الموضوع المعني في الوِب حتى تفهم معنى المصطلح وما يرمي إليه.
عليك كذلك التأكد أن المقابل الذي تستعمله يوصل المعنى المقصود وأنك
تستعمله في سياقه الصحيح فقد يكون له دلالات أخرى لا تقصدها. راعِ كذلك
قواعد الإملاء، و اسْعَ إلى تقليل اللَّبس عند القراءة بقدر الإمكان حتى
لو تطلب ذلك وقتا إضافيا في الصياغة و التشكيل و مراجعة الإملاء.

\subsection{وضوح المفاهيم}
تذكَّر أن الهدف هو أن يفهم الناس العاديون المصطلحات التي تستعملها. قد
يبدو صوغ مصطلحات تقنية يفهمها عامة الناس مستحيلا، لكن ثِق في قدرة الناس
على فهم مفردات لغتهم و تذكّر أن المصطلحات هي في واقعها مفردات من اللغة،
و هذا حقٌّ في العربية أكثر من لغات أخرى، فبها تزيد قدرة القارئ على
استنتاج دلالات مصطلحات لم يرها من قبل، أكثر من الإنگليزية مثلا. الناس
تعتاد المصطلحات الجديدة مع الوقت وكثير من الكلمات المعتادة اليوم، خاصة
في مجال المعلوماتية، كانت غريبة قبل بضع سنوات، حتى في الإنگليزية، و على
مَن هي لغتهم الأم.

المصطلحات القائمة المتداولة في مجالات العلوم و الهندسة والرياضيات
والإدارة ليست غريبة عن الحوسبة و المعلوماتية، و عموما ينبغي استعمالها
عند ملاءمتها فهي أقرب للفهم، بدلا من صوغ مصطلحات لم تستعمل من قبل، كما
يجب استعمالها عندما يكون موضوع البرمجية مرتبطا بأحد تلك المجالات.

\subsection{رَدُّ المصطلح إلى أصله}
أحيانا يكون من المفيد كون المصطلح في الترجمة واضحا كفاية بحيث يستطيع
المستخدم استنباط المصطلح الأصلي منه، خاصة عندما يكون المستخدمين معتادين
على المصطلحات الإنگليزية، أو عندما يحتاج المستخدم معرفة المصطلح باللغة
الإنگليزية ليبحث عنه في الوب أو في وثائق غير مترجمة للبرمجية.

مثلا: في حال كون مصطلحين مرتبطين في الإنگليزية مثل “log in/log out” أو
“enable/disable” أو “previous/next” فمن المفيد المحافظة على العلاقة في
الترجمة، مثل “ولوج\letterbackslash خروج” و “تفعيل\letterbackslash
تعطيل” و “السابق\letterbackslash اللاحق”.

\section[ref:37296725]{الاختصارات والأسماء والعلامات التجارية}
كثيرا ما تمر عليك أثناء الترجمة
اختصارات مثل “{\it e.g.}” و “{\it i.e.}” و “{\it SQL}” و “{\it FTP}”،
وأسماء منتجات مثل “{\it Firefox}” و “{\it OpenType}” و “{\it Google}” و
“{\it Windows}” و “{\it iPhone}”.

بعض تلك المصطلحات ليست جزءا من اللغة المحكية ولذا قد يبدو الحل الوحيد
لها تركها كما هي، أحيانا يكون هذا أفضل اختيار، لكن علينا الانتباه لبضعة
أمور.

قد يكون لبعض الاختصارات مقابلات متعارف عليها في اللغة الوجهة، مثل
الإنگليزية “etc.” التي تقابل في العربية “إلخ”. لكن بعض اللغات لا تعرف
الاختصارات المكتوبة مطلقا، كما أنه ليست لكل الاختصارات مقابلات في كل
اللغات، وسيكون عليك في هذه الحالة ترجمة المصطلح بغير اختصار، لكن إن كان
الاختصار اسما لتقنية ما (مثل “{\it XML}”) فمن الأفضل تركه كما هو إذ قد
لا يعرفه الجمهور المستهدف إلا بهذا الاسم المختصر، لكن في ترجمة
المستندات (مثل ملفات المساعدة أو صفحات الوب) حيث لا قيود على طول النص
(على عكس الحال في واجهات البرمجيات) تمكن إضافة ترجمة المصطلح لمزيد من
التوضيح، حسب الجمهور المستهدف، فلست تريد أن تكرر على قارئ متخصص ما
يعرفه سلفا.

قد تكون لبعض الشركات والمنتجات أسماء بلغتك تستعملها محليا، و عندها قد
يكون من الأنسب إيرادها. في المقابل بعض الأسماء علامات تجارية مسجلة وليس
مسموحا بتغييرها. لكن أحيانا تكون استثناءات خاصة للغات التي لا تكتب
بالأحرف اللاتينية، مثل العربية واليابانية، وفي هذه الحالة من الأفضل
السؤال للتأكد.

بعض أسماء المنتجات كلمات عادية وقد تكون ترجمة معانيها محبّذة، لذا إن كان
الغرض من الاسم أن يكون مفهوما ويعبر عن وظيفة البرمجية فمن الأفضل
ترجمته. الأدوات مثل برمجية الحاسبة ومحرر النصوص يشار إليها غالبا بكلمات
عامة دالة على الجنس وليس بأسماء مخصصة.

إن كنت ستنقل الاسم كما هو بالأحرف اللاتينية فانتبه إلى استعمال الأحرف
الكبيرة والصغيرة، ففي بعض أسماء المنتجات قد يكون نمط غير معتاد لحالة
الأحرف في أسماءها يشكل جزءا من العلامة التجارية، و المحافظة عليه تسهَّل
للقارئ التعرف على الاسم باللمح، خاصة إن كان لا يجيد قراءة الأحرف
الأجنبية، كأنه صورة.

\subsection[ref:20221205]{النقحرة}
إذا كانت لغتك لا تُكتب بالأحرف
اللاتينية (مثل العربية أو الصينية) فمن المهم وضع سياسة للنقحرة. هذه بعض
المعايير التي يمكن الاستعانة بها:

\startitemize[1]
\item إذا كان الاسم كلمة عادية تُوضِّح وظيفته أو دالة على جنس فالأفضل
ترجمته.
\item إذا كان الاسم لمنتج غير شائع أو مصطلح تقني قليل التكرار فقد يكون
من الأفضل تركه بالحروف اللاتينية كما هو ما دام لا يكثر وروده في
الترجمة.
\item إذا لم يكن للاسم معنى واضح فالأفضل تعريب كتابته صوتيا حسب النطق في
لغته الأم (لاحظ مثلا أن لا معنى لنقل اسم إسباني بالنطق الإنگليزي)، لكن
إن كان اختصارا ومنطوقه كلمة طويلة (مثل HTML) فقد يكون من الأفضل استبدال
كل حرف لاتيني بأقرب حرف له في لغتك، إما متصلين أو منفصلين بنقط أو
مسافات أو علامات فصل الأحرف إذا كان اتصال الأحرف يصعب القراءة أو يتشابه
مع كلمة في لغتك.
\stopitemize
\section{مُعينات ترجمة المصطلحات في أدوات الترجمة}
رغم أن وجود معجم جيِّد يفيد للغاية أثناء الترجمة، إلا أن تكرار النظر في
المعجم أثناء الترجمة قد تكون مرهقة، وفي بعض الأحيان لا يعلم المترجم متى
يراجع المعجم و قد يبني ترجمته على افتراضات خاطئة عن دلالات الكلمات.

إن كانت أداة الترجمة تدعم استخدام مسارد المصطلحات فسيساعد هذا المترجم
على النظر في مسرد المصطلحات كما قد تدفعه إلى مراجعة المصطلحات التي غفل
عنها. كما قد يساعد أيضا على إدراج ترجمة المصطلح دون الحاجة لكتابتها
وهذا قد يوفر الوقت ويتفادى الأخطاء الإملائية.

يمكن لبُوتِل استضافة مشروع مصطلحات على الخادوم يحوي مسارد مصطلحات تخص
مشروع ترجمة معين. طالع صفحة \at[ref:20355427] لمزيد من
التفاصيل.

يدعم فِرتال إظهار مقترحات للمصطلحات من عدة مصادر، بما فيها ملفات
المصطلحات في الحاسوب، و خدمات خاصة عبر الشبكة. انظر صفحة
\at[ref:36503815] لمزيد من المعلومات.

\chapter{إعادة الاستخدام}
يُنتِج المترجمُ أثناء عمله حصيلة قيمة من المصطلحات والترجمات التي يمكن
أن تفيد في ترجمات مشابهة. القدرة على الاستفادة لاحقا من هذا العمل عنصر
مهم في زيادة الإنتاجية، كما تضمن اتساق الترجمة بين المنتجات المختلفة أو
الإصدارات المختلفة من نفس المنتج.

\subsubsection{مثال}
يظهر هنا مثال على عرض الترجمات المشابهة للمترجم أثناء الترجمة.

\placefigure[here,force]
  {none}
  {\externalfigure[img002.png][width=.8\textwidth]}

يسمى حفظ الترجمات السابقة وإعادة استخدامها {\it ذاكرة ترجمة}. يختلف هذا
عن مسرد المصطلحات في أن ذاكرة الترجمة تتعامل مع جمل كاملة، كما أنها
تعرض اقتراحات على المترجم حتى لو لم تتطابق النصوص المصدر. فحتى لو كانت
كلمات برمتها في الاقتراح غير مناسبة، فإن اقتراحات ذاكرة الترجمة توفر
الكثير من الكتابة، كما تضمن توحيد بنية العبارات بين الترجمات المتشابهة.

\section{متى يُعاد استخدام الترجمة؟}
أي ترجمة موجودة ذات جودة مناسبة يجب اعتبار إعادة استخدامها. قد تفضل
استعمال الترجمات الأقرب، فمثلا إذا كنت تترجم أحد مشروعات گنوم فقد تفضل
ترجمات گنوم على ترجمات كدي، لكن في العادة لا تتباين أساليب الترجمة
كثيرا بين المشروعات المختلفة في أغلب اللغات وبالتالي ستستفيد من وجود كل
الترجمات السابقة في ذاكرة الترجمة.

يكون اللبس أكثر في النصوص الموجزة، فالرسائل من كلمة واحدة مثل “view” أو
“alert” قد تختلف ترجمتها حسب السياق، وحتى النصوص الأطول مثل “Empty
trash” ليست واضحة كفاية (قد تكون تقريرا لحال سَلّة المهملات الفارغة، أو
أمرا لإفراغ سلة المهملات)، لذا يجب الانتباه عند إعادة استخدام ترجمات
مثل هذه. أي إعادة استخدام لترجمات سابقة يجب مراجعتها للتأكد من مناسبتها
من ناحية الأسلوب والمصطلحات، وقد تساعد هذه المراجعة في اكتشاف أخطاء في
الترجمات السابقة وعليك الاستفادة من هذه الفرصة لتصويب الخطأ في مصدره.

\section{بعض المصادر المتاحة}
الترجمات السابقة للمشروع الذي تعمل عليه هي أهم مصدر لذاكرة الترجمة. كما
تستطيع جمع ترجمات الواجهة وملفات المساعدة وصفحات الوِب و غيرها معا
لتسريع العمل وضمان اتساقه.

تليها ترجمات البرمجيات ذات العلاقة، مثل برمجيات بيئة سطح المكتب ذاتها
التي تنتمي إليها (مثل گنوم مقابل كدي) وهي متاحة بسهولة، و سهولة الوصول
إلى العمل السابق هو أحد مميزات الكبيرة لتوطين البرمجيات الحرة.

تأتي في المرتبة التالية ترجمات البرمجيات المشابهة، فإذا كنت تترجم برمجية
للبريد الإلكتروني انظر ترجمات برمجيات البريد الإلكتروني الأخرى، أو إذا
كنت تترجم محرر صور فقد تمكنك الاستفادة من ترجمة عارض صور، وهكذا.

أخيرا تمكن الاستعانة بالخدمات التي تجمّع الترجمات:

\startitemize[1]
\item يجمّع Open-Tran.eu ترجمات العديد من مشروعات البرمجيات الحرة، ويوفر
موقعا للبحث فيها، كما تدعم استخدامه بعض أدوات الترجمة مثل فِرتال للبحث
عن اقتراحات.
\CURL{http://open-tran.eu} 
\item تُوفر ميكروسوفت أيضا موقعا لذاكرة الترجمة. لا تنسَ مراجعة هذه
الاقتراحات والانتباه إلى الاختلافات بين أسلوب الترجمة في ميكروسوفت
والأساليب والمصطلحات المتعارف عليها في البرمجيات الحرة.
\CURL{http://microsoft.com/language}
\stopitemize

\section{بعض الأساليب والأدوات}
تختلف الطريقة الأمثل للوصول إلى الترجمات السابقة من مشروع لآخر وقد تتغير
مع الوقت. هذه بعض الملاحظات المفيدة التي قد توفر عليك بعض الوقت:

\subsection{گنوم}
يمكن الوصول إلى ترجمات گنوم بسهولة عبر موقع
\URL{http://l10n.gnome.org} والمعروف باسم
“الأكاذيب اللعينة”. ابحث عن لغتك ثم أختر أيا من أطقم الإصدارات من
القائمة، يوجد رابط في أسفل الصفحة لتنزيل أحدث نسخة من كل الترجمات.

\subsection{كدي}
يمكن الوصول إلى ترجمات كدي بسهولة عبر موقع \URL{http://l10n.kde.org/}.
ابحث عن فريق ترجمة لغتك، وستجد رابطا في فقرة “الأرشيفات” لتنزيل ملفات
الترجمة.

\subsection{دِبيان}
يطوّر مشروع دبيان العديد من البرمجيات مثل أداة تنصيب دبيان وحزمة رموز
أيزو (تحتوي على ترجمات لأسماء الدول واللغات). يمكن الحصول على ملف يحوي
ذاكرة ترجمة كل المشروعات التي تطورها دبيان، من:
\CURL{http://i18n.debian.net/compendia}

كما تتوفر ترجمات غالبية الحزم التي يوزعها مشروع دبيان، في:
\CURL{http://i18n.debian.net/material/po}


من الأفضل الحصول على ملفات ترجمة كل مشروع مباشرة من موقعه، لكن الوصول
إليها جميعا هنا قد يكون أسهل وإن كان من المحتمل ألا تكون هي أحدث نسخة
من الترجمة.

\subsection{خواديم بوتل}
يُستخدم بوتل لإدارة ترجمة العديد من المشروعات ، منها أوبن​أوفس، ويمكن
الحصول على الترجمات من خادوم بوتل بسهولة.

اذهب إلى صفحة لغتك في مشروع ما على الخادوم، و منها إلى لسان “ترجم” وسترى
روابط لتنزيل ملف مضغوط يضم كل ملفات الترجمة. إذا لم تر هذا الرابط فقد
يتوجب عليك الولوج إلى الخادوم أولا أو طلب صلاحيات التنزيل من مدير
الموقع.

\subsection{موزيلا}
صيرورة التوطين الحالية في موزيلا تُصعِّب الوصول إلى الترجمات السابقة،
وإن كانت متاحة عبر Open-Tran.eu. لا تتوفر أي ذاكرة ترجمة رسمية، لكن
توجد ذواكر ترجمة غير رسمية منشورة بنسق TMX، وإن كانت المنتجات التي
تشملها ذواكر الترجمة هذه غير محددة، ولا تُعرفُ دقتها فيما يتعلق باللغات
التي لم تصدر رسميا أو أثناء مراحل تطوير الإصدارات المستقبلية من مشروعات
موزيلا:
\CURL{http://www.frenchmozilla.fr/glossaire/TMX}

إذا كنت تستخدم نسق PO في ترجماتك لمشروعات موزيلا (باستخدام عُدة الترجمة،
انظر صفحة \at[ref:20165030]) فيمكنك إعادة استعمال الترجمات من
ملفات PO بسهولة.

\subsection{تجميع ملفات الترجمة}
إذا كان لديك ملفات ترجمة تريد جمعها في ملف واحد وتحويلها إلى ذاكرة
ترجمة، فيمكنك استخدام بعض من أدوات عُدّة الترجمة. يمكن تخصيص أداة
\goto{pocompendium}[url(http://translate.sourceforge.net/wiki/toolkit/pocompendium)]
عبر عدة خيارات (مثل كيفية التعامل مع المسرعات) ويمكنك استخدام
\goto{po2tmx2}[url(http://translate.sourceforge.net/wiki/toolkit/po2tmx)]
لبناء ذاكرة ترجمة بنسق TMX الذي تستخدمه بعض أدوات الترجمة.

كما قد تجد في عُدّة الترجمة أدوات للتحويل من أنساق الترجمة الأخرى غير
نسق جِتتِكست الشائع (انظر صفحة \at[ref:20165030]).

\section{ذاكرة الترجمة و المصطلحات والترجمة الآلية}
{\it الترجمة الآلية} هي ترجمة يولدها الحاسوب آليا. عادة ما تعتمد هذه
الترجمة على قواعد للترجمة بين اللغات المختلفة أو على متن ضخم من
الترجمات السابقة. يمكن أن تكون مقترحات الترجمة الآلية غير طبيعية وغير
صحيحة وبها أخطاء إملائية ولغوية، لذا نقول بوضوح أنه {\bf لا يمكن
الاستعاضة بالترجمة الآلية عن المترجم الآدمي} ،لكنها وسيلة يمكن أن
تُساعد المترجم في عمله لاختصار بعض وقت البداية. بعض خدمات الترجمة
الآلية تعطي مقترحات مفيدة في حالات معينة، وإن كانت جودة الترجمة تختلف
كثيرا من لغة لأخرى ومن موضوع لآخر، فتحديد مدى ملائمة مُخرجات مترجم آليّ
ما راجع إلى تقدير الفريق.

في المقابل فإن الاقتراحات من ذاكرة الترجمة مصدرها دوما ترجمة سابقة
لمترجم آدمي، أيا كانت جودتها. قد تكون ترجمة نصٍّ مختلف و على برمجية
الترجمة إبراز الاختلافات بين المصدر الذي تترجمه والمصدر الذي بُني
الاقتراح عليه. يبرز فِرتال هذه الاختلافات ليساعد في الحكم على مدى
مناسبة الاقتراح وأي أجزائه ينبغي تعديلها.

لكن بينما قد لا تحتوي ذاكرة الترجمة دوما على اقتراح مناسب فإن الترجمة
الآلية بوسعها دوما اقتراح ترجمة لأي نص، حتى ولو لم يكن اقتراحا جيدا.
أحيانا يكون تحرير الترجمة الآلية لتصبح مناسبة أسهل من تحرير اقتراح من
ذاكرة الترجمة لا يتعلق كثيرا بالنص موضوع الترجمة. الأمر متروك لتقدير
المترجم ليحدد أي الاقتراحات أفضل وليصحح أي خطأ فيها.

كثيرون لا يثقون في جودة الترجمة الآلية، لذا يجب الانتباه عند التعامل
معها. من المهم أن يعي مستخدمو الترجمة الآلية قصورها وأن الاعتياد على
تلك الترجمة قد يقلل من قدرتهم على اتباع الأساليب اللغوية الصحيحة، لكن
في ذات الوقت قد تكون الترجمة الآلية عونا كبيرا خاصة لمن لا يستطيع
الكتابة سريعا.

\section{إعادة الاستخدام في فِرتال}
يستطيع فِرتال عرض مقترحات من أكثر من مصدر، بما فيها الملف الحالي و
الملفات التي سبق تحريرها وحفظها، ومن بعض الخدمات على الشبكة. بعض تلك
المصادر الشبكية هي ذواكر ترجمة وبعضها ترجمات آلية، لذا تنبغي معرفة نوع
كل مصدر و خصائصه لتقيّم اقتراحاته. انظر صفحة
\at[ref:36543315] لمزيد من المعلومات.

\chapter[ref:32352020]{مسائل تقنية}
تختلف ترجمة البرمجيات عن ترجمة
الوثائق، ويتناول هذا القسم بعضا من المسائل الشائعة التي على المُوَطِّن
الإلمام بها. قد تبدوا بعض هذه المسائل معقدة لأول وهلة، لكن لا تقلق
فستعتادها بسرعة كما أن أغلب الترجمات تخلو من هذه المسائل.

\section{ما لا يُترجم}
من الشائع وجود عناصر في الترجمة تترك كما هي، مثل {\bf عناوين البريد
الإلكتروني} و {\bf مسارات مواقع الوب}، فينبغي التأكد من أن أعضاء الفريق
يعلمون كيف تبدو عناوين البريد و مسارات ومواقع الوب. لكن هناك استثناءات،
مثل عناوين الوب أو المسارات التي تشير إلى مواقع أو وثائق بعدة لغات، ففي
هذه الحالة يمكن تغيير المسار ليشير إلى الإصدارة باللغة المستهدفة، إن
وُجدت.

ذكرنا من قبل {\bf العلامات التجارية} و {\bf أسماء المنتجات} في صفحة
\at[ref:37296725]. تأكد من إلمامك بالطريقة المثلى للتعامل مع
هذه الأسماء في ترجمتك.

وهناك أيضا {\bf الدوال البرمجية} وغيرها من جوانب بروتوكولات الشبكات أو
لغات البرمجة. يُذكر أحيانا في رسائل الأعطال اسم الأمر الذي نتج عنه
العطل، وهذه الأوامر يجب في الأغلب الأعم ألا تترجم\footnote{بعض اللغات
تترجم فيها الدَّوَال الإحصائية و الرياضية في تطبيقات جداول العمل
الممتدة، وعليك إن تعرف إن كان هذا مطبقا في لغتك في التطبيق موضوع
الترجمة.}. مثلا قد يشير تطبيق جداول العمل الممتدة إلى الدالة \Type{COUNT()}
(مدخلة في أحد خلايا الجدول لحساب المجموع). نرى تلميحان يشيران إلى أنها
دالة؛ كونها مكتوبة بأحرف لاتينية كبيرة، وزوج الأقواس في نهايتها. وضع
كلمة بين علامتين تنصيص قد يكون أيضا إشارة إلى أنها لا تترجم.

\subsubsection{مثال}
\startitemize[1]
\item {\tt مُرِّرَ مجالٌ غير صحيح إلى \ltr{\bf COUNT()}.}
\item {\tt خطأ نحوي في {\bf SELECT}.}
\item {\tt عدد معاملات غير صحيح إلى \ltr{\bf printf()}.}
\item {\tt انتهت مهلة الشبكة أثناء {\bf POST}.}
\item {\tt قيمة “{\bf errno}” غير صحيحة.}
\stopitemize
بالطبع قد توجد أسباب أخرى لكتابة كلمة بأحرف لاتينية كبيرة أو ضعها بين
علامتي تنصيص.

مفهوم آخر قد يكون غريبا على مترجمين عديدين هو {\bf خيارات سطر الأوامر}.
فعند تشغيل برمجية بكتابة اسمها في سطر الأوامر (بدلا من النقر على أيقونة
أو في قائمة) فمن الممكن عادة تمرير خيارات إضافية إليه للتحكم في مسلكه،
و بالرغم من أن هذا محجوب عن معظم المستخدمين فإن وثائق الاستخدام قد تشير
إلى هذه الخيارات. يجب ألا تترجم هذه الخيارات لأن البرمجية لن تتعرف
عليها إلا في نصّها الأصلي (إلا إن كانت مُوطنّة في البرمجية ذاتها)، لذا
يجب أن تتضمن الوثائق المترجمة الخيارات بنصِّها الأصلي.

\subsubsection{مثال}
\rexample{لمعرفة رقم إصدارة البرمجية شغّلها مع \ltr{\bf --version}.}

في حال كانت الخيارات تقبل تمرير {\it مُعامِلات} إليها، فيمكن ترجمة
الكلمة التي تشير إلى المُعامِل لأنها في العادة ليست إلا مثالا يوضح نوع
المعامل و معناه.

\subsubsection{مثال}

\starttable[|r|r|]
\HL
\NC \ReFormat[cB]{المصدر} \NC \ReFormat[cB]{الترجمة}\SR
\HL
\NC \ltr{\type{--playlist=}\bf PLAYLIST}
\NC \ltr{\type{--playlist=}\bf قائمة استماع}\FR
\NC \ltr{\type{--add-to=}\bf ARCHIVE}
\NC \ltr{\type{--add-to=}\bf أرشيف}         \LR
\HL
\stoptable

جرت العادة في بعض مشروعات البرمجيات على كتابة المُعامِلات بأحرف لاتينية
كبيرة مما يساعد على تمييزها. ترجمة مُعامِلٍ (مثل “PLAYLIST”) تنبه
المستخدم إلى أن الخيار (مثل “\Type{--add-to}”) يجب أن يترك كما هو في حين
عليه التعويض عن اسم المعامل بقيمة مناسبة. من المعتاد أيضا عند الإشارة
إلى المُعامِل في مواضع أخرى من الوثائق أن يكتب بأحرف كبيرة، وفي اللغات
التي لا توجد بها حروف كبيرة يمكن وضعه بين علامتي تنصيص لتمييزه عن باقي
النص، أو تمييزه بصريا بوسيلة أخرى، مثل الأحرف المائلة إن أمكن.

\section{الخُلوص}
الخُلُوص وسيلة تقنية لإدراج محارف حاسوبية خاصة في النصوص؛ محارف ذات
وظائف خاصة تمنع إدراجها في النصوص على النحو المعتاد بمجرد إدخالها من
لوحة المفاتيح. يتألَّف تتابع الخلوص من علامة خلوص متبوعة بحرف أو أكثر
من المحارف المحجوزة للاستخدامات الخاصة. مع أن المترجمين لا يحتاجون
للتعامل مع هذا كثيرا إلا أن من الأفضل معرفته. أكثر الخلوص شيوعا الشرطة
المائلة إلى الأمام (\type{\}) متبوعة بحرف واحد. في الأغلب
سيوضح النص المصدر موضع هذه العلامات ليضعها المترجم في الموضع المقابل في
الترجمة.

\subsection{أنواع محارف الخلوص}
%لهذه {\it المحارف المُخلّصة} معانٍ خاصة:

\placetable
  [here,force,nonumber]
  {لهذه {\it المحارف المُخلّصة} معانٍ خاصة}
\starttable[|l|lp(.2\textwidth)|lp(.6\textwidth)|]
\HL
\NC \bf تتابع الخلوص \NC \bf أثره \NC \ReFormat[cB]{وصفه}\SR
\HL
\NC \Type{\n}  \NC سطر جديد \NC
يبدأ سطرا جديدا (مماثل لضغط زر الإدخال).
قد تمثِّله أداة الترجمة بالرمز \type{¶}.\FR
\NC \Type{\t}  \NC جدولة \NC
يُدخل علامة جدولة في هذا الموضع من النص.
يستخدم هذا أحيانا لمحاذاة النصوص في أعمدة.\MR
\NC \Type{\r}  \NC الرجوع (إلى أوّل السطر، كالآلة الكاتبة) \NC
عادة ما يُزاوج هذا مع \Type{\n} في نظام وندوز عند كل سطر
جديد، بينما يُستخدم بدلا منه في نظام ماكنتوش.\MR
\NC \Type{\\}  \NC الشرطة مائلة للأمام ذاتها \NC
لإدراج الشرطة المائلة للأمام بذاتها، فبما أنها محرف مخصص للخلوص،
فطباعتها تتطلب خلوصا.\MR
\NC \Type{\"} \NC علامة تنصيص مزدوجة (") \NC
في بعض الحالات تكون لعلامة التنصيص المزدوجة معنى خاصا،
وينبغي تخليصها إن أردنا إدخال علامة التنصيص ذاتها.\MR
\NC \Type{\uxxxx}  \NC محرف خاص \NC
أي محرف أو علامة خاصة مثل © أو ™.
أحيانا يَستخدِم المبرمجون \Type{\u} للإشارة إلى أي من المحارف الخاصة
غير المعتاد طباعتها من لوحة المفاتيح، حيث \type{xxxx} هنا هو
رقم عشري أو سِتَّعَشريّ يدل على رمز المحرف الخاص في يونيكود، وفي الغالب
لا حاجة لتغييره.\MR
\NC \Type{&apos;}  \NC علامة تنصيص مفردة (') \NC في بعض مستندات XML ترد علامة
التنصيص المفردة وغيرها من العلامات باستخدام ما يسمى كيانات XML. مثل:

{\tt <a href="index.html" title="Rock {\bf \&apos;}n Roll">مربط</a>}\LR
\HL
\stoptable

\subsubsection{مثال}
\starttable[|rp(.2\textwidth)|p(.4\textwidth)|p(.35\textwidth)|]
\HL
\NC \ReFormat[cB]{ما تراه}\NC\ReFormat[cB]{معناه}\NC\ReFormat[cB]{ما تفعل}\SR
\HL
\NC …server active.{\bf\tt\letterbackslash n} Do you want…  \NC
سطر جديد بين جملتين. عند تشغيل البرمجية ستظهر كل جملة
في سطر وحدها.
\NC أدخل \Type{\n} في نفس الموضع بين الجملتين في الترجمة.\FR
\NC … print jobs or{\bf\tt\letterbackslash n}other actions… \NC
هذه يصعب تمييزها إذ سيوجد سطر جديد بين “or” و “other”.
\NC تمعَّن في بنية الجملة كلها. هنا غالبا يُقصد بالسطر الجديد
توزيع النص على السطور بتساوٍ، فقطِّ أسطر الترجمة بما يحقق ذات الغرض و
يناسب اللغة المستهدفة.\LR
\HL
\stoptable

عادة يتوجب نسخ محارف الخلوص كما هي دون تغيير، ولا تُغفِل التأكد من أن
عدد تتابعات الخلوص الذي أدخلت في الترجمة يساوي عددها في المصدر. إذا
كانت لغة الوجهة لا توظِّف بعض هذه الأحرف، مثل علامات التنصيص، أو تستعمل
بدائل لها فيمكنك إما إهمالها أو وضع البديل المناسب للغة، الأقواس
المحدبة مثلا.

لا تغير تتابع الخلوص دون سبب، فلا تضع مثلا {\bf\tt /n} بدلا
من {\bf\tt\letterbackslash n} لمجرد أنك لا تجد \type{\}  على لوحة
المفاتيح. انتبه للحالات مثل “\type{File \ Open \ New}” فهذا ليس خلوصا،
وعموما كل ما ليس في الجدول أعلاه فهو في الأغلب ليس من الخلوص.
انتبه كذلك لنقل توزيع المسافات ذاته حول تتابعات الخلوص.

\section{الرَّقْم}
يُستخدمُ HTML لإنشاء صفحات الوب، كما سيمر عليك كذلك XML (الذي يشبه HTML
كثيرا) في ترجمات واجهات البرمجيات. كلاهما مثال على لغات الرَّقْم التي
تُستخدم لهيكلة النصوص بنسق معين. انتبه إلى مقاطع النصوص في HTML التي
يمكن ترجمتها والمقاطع التي يجب أن تُترك كما هي لأنها جزء من لغة
الرَّقْم ذاتها. تتميز HTML بوسومها المحفوفة بقوسين زاويين، مثل:

\example{<tag>}

كما يمكن أن يحتوي الوسم على معلومات إضافية:

\example{<img src="picture.jpg" />}

يحتوي وسم img (صورة) هنا على خصيصة إضافية تحدد اسم ملف الصورة. وقد تكون
الوسوم في أزواج كذلك:

\example{<p> </p>}

لاحظ الفرق بين وسم الفتح ووسم الغلق.

\subsection{ما لا يُترجَم}
لا تترجم الوسوم ذاتها. تتألَّف HTML من وسوم تحدد بداية ونهاية مقاطع من
النص، قد يكون النصَّ عنوانا أو فقرة أو رابطا أو حتى جزءًا من النص سيعرض
بتنسيق مغاير:

\startitemize[1]
\item {\tt{\bf <h1>}عنوان{\bf </h1>}}
\item {\tt{\bf <p>}فقرة في النص{\bf </p>}}
\item {\tt{\bf <a href=bob.html>}مربط إلى صفحة{\bf </a>}}
\item {\tt نص عادي و {\bf <b>}نص عريض{\bf </b>}}
\stopitemize
بعض الوسوم تغري بترجمتها، مثل \type{<title>} و \type{<center>}
و \type{<body>}، لكن تجب مقاومة الإغراء ولا تترجمها، و إلا لن تعمل.

\subsection{تغيير الوسوم أو ترتيبها}
يكون من الضروري أحيانا تغيير ترتيب الوسوم، تبعا لتغيّر ترتيب الجملة
المترجمة عن ترتيب المصدر.

بعض الوسوم تمثل عناصر ينبغي ظهورها في مواضع معينة في الترجمة، مثل صورة
أو هامش، ويجب وضعها في الترجمة لتعطي نفس التأثير كما في المصدر.

تأكد دوما أن وسوم الفتح والغلق تؤلف أزواجا متكاملة، وأن وسوم الغلق تأتي
بعد وسوم الفتح.

\subsection{أي الخصائص تُتَرجَم}
الخصائص معلومات إضافية في متن الوسم.

\subsubsection{مثال}
\example{<body bgcolor=blue>}

فهنا “bgcolor” خصيصة قيمتها “blue”. الخصائص مثلها مثل الوسوم لا تُترجم،
لكن بعض القيم تنبغي ترجمتها. القيمة “blue” في المثال السابق يجب ألا
تترجم.

بعض الخصائص ينبغي أن تترجم قيمها:

\starttable[|l|p(.8\textwidth)|]
\HL
\NC {\bf الخصيصة} \NC \ReFormat[cB]{معناها}\SR
\HL
\NC \type{alt}  \NC وصف نصي بديل للصورة، أحيانا يؤلف جزءا من وسم img (صورة)،
وهو مهم للأشخاص المعاقين أو للمتصفحات النصية و لتيسير البحث.\MR
\NC \type{title}  \NC عنوان يظهر عندما يمر مؤشر الفأرة على شيء ما، يشبه تماما
بالونات تلميحات الأدوات في واجهة المستخدم.\LR
\HL
\stoptable

في بعض الأحيان قد تحتاج لتغيير صفتي “lang” أو “dir”، لكن من الأفضل ترك
هذا لمن لهم خبرة في استخدام HTML.

\subsection{نصوص تشبه الوسوم لكن تنبغي ترجمتها}
أحيانا تكتب بعض العبارات بحيث تبدو كالوسوم لكنها ليست في الحقيقة وسوما،
وتنبغي ترجمتها، مثل <Error> أو <File not found> إلخ.

إذا كانت لك خبرة في HTML فستميز الوسوم الحقيقية من غيرها بسهولة. إذا كنت
لا تعرف فاتبع هذه القاعدة: إذا كانت كل الحروف صغيرة أو كل الحروف كبيرة
فهذا على الأغلب وسم، أما إن كانت مختلطة فغالبا ليس وسما، إن كان يتضمن
خصائص (مثل \type{<font color=blue>}) فمؤكد أنه وسم، وأخيرا إذا كانت هي الكلمة
الوحيدة في الجملة (مثل “<None>”) فغالبا ليست وسما.

\section{الجموع الساذجة}
تُصاغ أغلب الجموع في اللغة الإنگليزية بإلحاق الحرف {\bf s} بآخر الكلمة،
وستجد كثيرا من المبرمجين يضيفون {\bf (s)} إلى الكلمة للدلالة إلى أنها
قد تكون مفردا أو جمعا حسب السياق و العدد قبلها. في اللغة العربية (على
سبيل المثال) يصاغ جمع التكسير بتغيير أحرف الكلمة، فقد تضاف حروف أو تحذف
من أي موضع في الكلمة، فلا يمكن اتبّاع النسق ذاته كما في الإنگليزية
للجمع.

\example{Show/Hide Axis Description{\bf (s)}}

\rexample{أظهر\letterbackslash أخفِ (أ)وص(ا)ف المحور}

علاوة على أن صياغة مفردات أخرى في الجملة قد تتغير حسب الإفراد أو التثنية
أو الجمع ولا يمكن تمثيل كل الحالات بصياغة واحدة.

هذه بعض خيارات التعامل مع هذا النوع من الجموع:

\startitemize[n]
\item إذا لم تكن في لغتك جموع فترجم الجملة متجاهلا صيغ الجمع.
\item تواصل مع المبرمج للتباحث في إمكانية تطبيق أسلوب الجموع الصائبة
(انظر صفحة \at[ref:33482409]).
\item استعمل بُنية مماثلة لجموع الإنگليزية إن كان هذا يناسب لغتك.
\item تجاهل الصيغة المفردة وترجم الجملة كما لو كانت جمعا. هذا ما يُستعمل
في اللغات التي تختلف فيها صيغة الإفراد عن الجمع اختلافا كبيرا ولا يمكن
فيها استعمال طريقة اللواحق\letterbackslash البوادئ في الأقواس.
\stopitemize

\section[ref:33556809]{المتغيرات}
المتغيرات رموز تحجز الموضع لعدد أو
نص آخر يعوّض عنها به أثناء تشغيل البرمجية. لتستطيع البرمجيات عرض رسالة
مثل “There are 3 files remaining” تحتاج إلى رسالة مُعمَّمة الصيغة يمكن
استعمالها لعرض “There are 4 files remaining” و “There are 5 files
remaining” إلخ. وقد تبدو تلك الرسالة كالتالي:

\example{There are {\bf \%d} files remaining}

الرمز “\Type{%d}” هو المتغير هنا ووظيفته حجز موضع سيشغله العدد لاحقا. توجد
صور أخرى تؤدي النتيجة ذاتها:

\startitemize[1]
\item {\tt There are {\bf \{\}} files remaining}
\item {\tt There are {\bf \%(number)d} files remaining}
\item {\tt There are {\bf \%1} files remaining}
\item {\tt There are {\bf \&number;} files remaining}
\item {\tt There are {\bf \$n} files remaining}
\stopitemize

الصيغ السابقة كلها متماثلة وستنتج عنها عند تشغيل البرمجيةِ النتيجةُ
ذاتها، شيئ مثل “There are 2 files remaining”. في كل تلك الأمثلة يمثل
المتغير عددا يكتب بالأرقام، لا بالكلمات مثل “There are two files
remaining”. اترك المتغير كما هو ولا تحاول ترجمة “number” في المثالين
الثاني والرابع.

لا تُستخدم المتغيرات للأعداد و حسب، ففي المثال التالي يستخدم المتغير
ليحجز موضعا لاسم التطبيق:

\example{Open with “{\bf \%s}”}

ستظهر شيئا مثل:

\example{Open with “OpenOffice.org”}

أو

\example{Open with “Virtaal”}

أحيانا يضع المبرمجون تعليقا يوضح ما يمثله المتغير و أية تغيير مطلوب. إذا
كان اسم المتغير ذا دلالة (مثل “number” في الأمثلة السابقة) فسيبين لك
نوعه. من المفيد تذكُّرِ أن “\Type{%d}” يمثل دوما عددا، بينما يمثل “\Type) كبداية لأسماء
المتغيرات، و هي جزء من اسم المتغير لذا لن تظهر عند تشغيل البرمجية، فإذا
رغبنا في إظهار رمز النسبة المئوية في الترجمة فعلينا كتابة “\type” لتظهر
علامة نسبة مئوية واحدة أثناء تشغيل البرمجية.

\subsubsection{مثال}
\example{The download is \%d{\bf \%\%} complete.}

ستظهر هذه الرسالة نصا مثل “The download is 40\% complete” أو “The
download is 100\% complete”. عليك الانتباه إلى أن “\Type{%d}” متغير لقيمة
عددية (مثل 40 أو 100 في المثال السابق) بينما تمثل “\Type” علامة نسبة
مئوية واحدة في الترجمة.

لا تقلق إذا لم تعرف كيف تتعامل مع هذه الحالات فهي نادرة وتستطيع سؤال
مطوري المشروع المساعدة حينها.

بعض الأمور التي ينبغي الانتباه إليها:

\startitemize[1]
\item يجب عدم ترجمة اسم المتغير نفسه أو تغييره على أي نحو إلا إن كنت
متأكدا مما تفعله. يجب أن تكون قادرا على تمييز المحارف التي تؤلِّف اسم
المتغير من النص المحيط بها.
\item عليك وضع المتغير في الموضع المناسب له في الترجمة لتظهر الرسالة
المترجمة طبيعية حسب قواعد اللغة، وقد تحتاج لضبط نص الترجمة ليناسب ما
سيُعوَّض به عن المتغير، فلا تتبع ترتيب الكلمات في النص المَصدَر حرفيًا.
\item إذا تعددت المتغيرات فقد يتطلّب الأمر تغيير ترتيبها. المتغيرات ذات
الأسماء يمكن ترتيبها بلا قيد، لكن المتغيرات الرمزية (مثل “\Type{%d}” أو
“\Type{%s}”) تتطلب معاملة خاصة وأحيانا لا يمكن تغيير ترتيبها عمّا في
المصدر.
\item أحيانا ما تشير المتغيرات إلى أشياء غير معتادة مثل بعض رسائل
الأعطال غير المترجمة أو أسماء الملفات، لذا قد ترغب في إحاطتها بعلامتي
تنصيص إذا كانت هذه طريقة مناسبة لتمييز النصوص غير المعتادة في لغتك.
\stopitemize

توجد مسائل أخرى ثانوية متعلقة بالمتغيرات، طالع قسم “التعامل المتقدم مع
المتغيرات” في صفحة \at[ref:43252826]، وكذلك هذه المقالة:
\CURL{http://translate.sourceforge.net/wiki/guide/translation/variables}


\section[ref:33482409]{الجموع الصائبة}
للبرمجيات القدرة على التعامل
الصحيح مع صيغ الجمع. تحدد البرمجية أثناء تشغيلها أي صيغ الجمع يُستعمل
بناء على العدد الذي سيُعرض، مثل قواعد المعدود في العربية. فمثلا سيعرض
بالإنگليزية:

\startitemize[1]
\item \ltr{0 files}
\item \ltr{1 file}
\item \ltr{2 files}
\item \ltr{3 files}
\item \ltr{100 files}
\item \ltr{109 files}
\item إلخ.
\stopitemize

فكما ترى تغيَّر التمييز طبقا لعدد العناصر. في أغلب اللغات يختلف تمييز
العدد طبقا للعدد وعليك معرفة القاعدة المطبقة في لغتك بدقة. في
الإنگليزية، توجد حالتان؛ حالة للمفرد (“\ltr{1 file}”) وأخرى للجمع (“files”،
أي رقم عدا 1).

\subsubsection{مثال}
\startitemize[1]
\item \ltr{\tt{\bf \%d file was} added to the archive.}
\item \ltr{\tt{\bf \%d files were} added to the archive.}
\stopitemize

لاحظ الاختلاف في تمييز العدد بين الحالتين وفي بنية الجملة كذلك (“was” و
“were”).

في اللغات التي ليست فيها صيغة للجمع ستُترجم كل من الجملتين بالجملة
المقابلة ذاتها، وهذا شائع في بعض اللغات الآسيوية. في اللغات التي فيها
قاعدة مطابقة للإنگليزية ستُترجم كل جملة بجملة مقابلة لها (واحدة للمفرد
وواحدة للجمع). أما في اللغات التي فيها أكثر من حالتين فستتضمَّنُ
ترجمَتُها جملةً لكل حالة حسب العدد طبقا للقواعد اللغوية.

دائما ما تحتوي صيغ الجموع الصائبة على متغيرات (انظر صفحة
\at[ref:33556809]) لأن العدد لا يكون معلومًا سلفًا وقت صياغة
العبارات، ففي المثال السابق يُمَثَّل المتغير “\Type{%d}” عددا مجهولا. انتبه
دوما لما يُمَثِّله المتغير ولا تنسَ ترتيب الجملة لتكون اللغة طبيعية.

قد يطلب منك المبرمجون –في حالات قليلة– معلومات إضافية عن قواعد التعامل
مع صيغ الجمع في لغتك. قد تجد المعلومات المطلوبة في هذه الصفحة:
\CURL{http://translate.sourceforge.net/wiki/l10n/pluralforms}

\section[ref:34484726]{المُسرِّعات}
المُسرِّع (“accelerator” ويسمى
أحيانا “hot key” أو “access key” أو “mnemonic”) هو زر في لوحة المفاتيح
تضغط عليه لتفعيل قائمة أو إصدار أمر ما بسرعة. إذا نظرت إلى شريط القوائم
في أي تطبيق فستجد خطًا تحت بعض الحروف في عناوين القوائم. لاحظ الخط تحت
الحرف الأول في كلٍ من \underbar{F}ile و \underbar{E}dit و
\underbar{V}iew. لفتح قائمة ملف تضغط \KBD{Alt+F}. ستلاحظ كذلك أن أغلب عناصر
القائمة لها مُسرِّعات، فللوصول إلى \underbar{O}pen تضغط {\bf o} بعد
الضغط على \KBD{Alt+F}.

علامة المسرع هي رمز خاص نستعمله لتعيين حرف المُسَرِّع في الترجمة.

\subsubsection{التعرف على المسرع}
تتعدد علامات المسرعات وتختلف من برمجية\letterbackslash نظام لآخر:

\starttable[|lp(.15\textwidth)|l|l|l|l|lp(.15\textwidth)|]
\HL
\NC {\bf البرمجية} \NC {\bf العلامة}        \NC
    {\bf الاسم}    \NC {\bf النص الإنگليزي} \NC
    {\bf كيف يظهر} \NC {\bf ملاحظات}        \SR
\HL
\NC كدي و Qt و wxWidgets \NC \type{&} \NC علامة \& \NC \ltr{\tt Save {\bf \&}As…} \NC \ltr{Save \underbar{A}s…}\NC \FR
\NC گنوم و GTK \NC \type{_} \NC الشرطة على السطر \_ \NC \ltr{\tt Save {\bf \_}As…} \NC \ltr{Save \underbar{A}s…}\NC \MR
\NC أوبن​أوفِس \NC \type{~}  \NC علامة \lettertilde  \NC \ltr{\tt Save {\bf \lettertilde }As…}  \NC \ltr{Save \underbar{A}s…} \NC \MR
\NC موزيلا \NC \type{&}  \NC علامة \& \NC \ltr{\tt Save {\bf \&}As…}  \NC \ltr{Save \underbar{A}s…} \NC عند استخدام \goto{moz2po}[url(http://translate.sourceforge.net/wiki/toolkit/moz2po)]
\MR
\NC ملفات \goto{.rc}[url(http://translate.sourceforge.net/wiki/toolkit/rc)] في وِندوز  \NC \type{&}  \NC علامة \& \NC \ltr{Save {\bf \&}As…}  \NC \ltr{Save \underbar{A}s…} \NC تستخدم كذلك في واين\LR
\HL
\stoptable

الضغط على {\bf A} في أي من الأمثلة السابقة سيفتح نافذة الحوار Save As.

ماذا لو أردنا إيراد الرمز \type{&} دون أن يكون علامة على مسرع؟ في هذه الحالة
سيُكتب بطريقة خاصة؛ \type{&&} (في كدي) أو \Type{&amp;} (في موزيلا): “\type{Mail
 && News}” أو “\type{Mail &amp; News}”. يُمكن بالطبع عند الترجمة تجاهل \type{&} ووضع
الكلمة المقابلة في لغتك، “وَ” مثلا.

\subsubsection{اختيار المسرع}

\placetable
  [nonumber]
  {أمثة}
\starttable[|lp(.15\textwidth)|lp(.15\textwidth)|lp(.15\textwidth)|lp(.15\textwidth)|lp(.15\textwidth)|lp(.15\textwidth)|]
\HL
\NC {\bf النص الإنگليزي} \NC {\bf ترجمة جيّدة}
\NC {\bf ترجمة سيئة} \NC {\bf شكل المصدر} \NC
\ReFormat[c]{\bf شكل الترجمة} \NC \ReFormat[c]{\bf ملاحظات}\SR
\HL
\NC \Type{&File} \NC \&ملف \NC م\&لف \NC \underbar{F}ile \NC \underbar{م}لف
\NC قاعدة 1 و 2\FR
\NC \Type{&Jump} \NC ت\&نقّل \NC \&تنقّل \NC \underbar{J}ump \NC ت\underbar{ن}قل \NC
قاعدة 3 و 4\MR
\NC \Type{~Tools}  \NC Di\lettertilde tlabela \NC \lettertilde
Ditlabela \NC \underbar{T}ools \NC Di\underbar{t}labela \NC قاعدة 1 و
3\MR
\NC \Type{&Help} \NC T\&hušo \NC \ltr{\&Thušo} \NC \underbar{H}elp \NC
T\underbar{h}ušo \NC قاعدة 1\MR
\NC \Type{_Move} \NC Š\_uthiša \NC \ltr{\_Šuthiša} \NC \underbar{M}ove \NC
Š\underbar{u}thiša \NC قاعدة 8\MR
\NC \Type{Zoom &100%} \NC Godiša \&100\% \NC \ltr{\&Godiša 100\%} \NC Zoom
\underbar{1}00\% \NC Godiša \underbar{1}00\% \NC قاعدة 11\MR
\NC \Type{~Font Size} \NC Bogola bja \lettertilde fonte \NC Bogola
bja\lettertilde fonte \NC Font Size \NC Bogola bja \underbar{f}onte \NC
المسافات\LR
\HL
\stoptable

بعض القواعد التي قد تفيد في اختيار المسرع المناسب:

\startitemize[n]
\item عيِّن الحرف الأبرز، مثل فاء مصدر الكلمة، من الكلمة الأبرز في
العبارة.
\item تجنَّب الأحرف التي يقلل وجود شرطة المسرع تحتها من مقروئيتها.
\item تجنّب تكرار المسرعات. ينبغي ألا يُعيَّنَ الحرف ذاته أكثر من مرة
كمسرّع في القائمة ذاتها أو النافذة أو المربع الحواري ذاتهما، و مع أن
هذا لا يمنع تشغيل البرمجية إلا أن المسرع لن يعمل كما ينبغي فتَقِلَّ
سهولة الاستخدام.
\stopitemize

عادة ما نتبع هذه القواعد البسيطة:

\startitemize[n]
\item حاول تعيين الحرف الأبرز في الكلمة: “\Type{&Open}” ← “ا\&فتح”.
\item حاول أن يكون في نفس الموضع من الكلمة المصدر: “\Type{&File}” ← “\&ملف”.
\item عيِّن حرفا غير شائع، لتقليل احتمال التعارض مع مسرعات أخرى.
\item حاول تعيينَ حرفٍ موضعه على لوحة المفاتيح قريب من الحرف في اللغة
المصدر، ليسهل على المستخدمين اعتياده.
\item تجنّب الأحرف الشاقولية مثل {\bf ا} فقد تصعب في بعض الخطوط رؤية
الشرطة تحتها لقصرها.
\item تجنّب الأحرف النازلة عن السطر مثل {\bf ح} (مفردة أو في آخر الكلمة)
أو {\bf ق} أو {\bf و}، إذ قد تتداخل الشَّرطة مع الحرف فيصعب تمييزه.
مثلا في كدي \&ح قد تظهر مثل \underbar{ح}. من الأفضل التأكد من هذا
باختبار النتائج في البرمجية بصريا، فبعض النُّظُم تتعامل جيدا مع هذه
الحالة.
\item تجنَّب الأحرف التي تجاور أحرف نازلة عن السطر للسبب السابق ذاته.
\item تجنّب الأحرف غير الظاهرة على لوحات مفاتيح أغلب المستخدمين، مثل
{\bf ڤ} أو {\bf پ} أو {\bf گ}.
\item تجنَّب الأحرف المنقوطة من أسفل مثل {\bf ب} أو {\bf ي}، للسبب
أعلاه.
\item إن كانت لغتك تستعمل طريقة إدخال متقدمة (مثل بعض اللغات الآسيوية)
فلا يمكن تعيين أي من أحرف اللغة الوجهة، وفي هذه الحالة يمكن تعيين
المسرع الذي في النص المصدر (الإنگليزية): “\Type{&File}” ← “\Type{XXXX(&F)}”.
في بعض البرمجيات يمكن ترك المسرع دون تعيين وعندها سيُعيَّن ما في
المَصدَر تلقائيا.
\item يمكن تعيين مسرعات غير الأحرف مثل الأرقام وعلامات الترقيم ما دامت
موجودة في لوحة المفاتيح الشائعة بين متحدثي لغتك، وخاصة إذا كان ذلك هو
المُعيِّنُ في النَّص المَصدَر.
\stopitemize

\subsubsection{التعارض بين المسرعات}
عند وجود مسرعين مختلفين يُعيِّنان الحرف ذاته يعد ذلك تعارضا، بفرض أن
العناصر التالية هي أول ثلاث مدخلات في قائمة:

\startitemize[1]
\item \underbar{م}لف
\item \underbar{م}ستندات
\item وثائق ال\zwj\underbar{م}ساعدة…
\stopitemize

كما يظهر فإن المسرعات الثلاثة تُعيِّن الحرف {\bf م}، وهذا تعارض. لحسن
الحظ فأغلب التطبيقات ستختار العناصر بالتتالي إذا كررت ضغط {\bf م}.

وهذا تعيين أفضل للمسرعات في تلك القائمة:

\startitemize[1]
\item \underbar{م}لف
\item مس\underbar{ت}ندات
\item و\underbar{ث}ائق المساعدة…
\stopitemize

هكذا نُعيِّن {\bf م} و {\bf ت} و {\bf ث} فلا يحدث تعارض.


\subsubsection{المراجعة}
اختبر تعييناتك بصريا بتشغيل البرمجية. التعارض ليس مشكلة كبيرة وغالبا
سيُحل مع الوقت. راجع قسم “المراجعة العملية” صفحة
\at[ref:34365426].

\section[ref:43252826]{التعامل المتقدم مع المتغيرات}
أحيانا يتعرَّض المترجم للتعامل
المتقدم مع المعاملات لتطويع كيفية عرضها. الأمثلة والأساليب في هذا القسم
متقدمة قليلا، ولا يُشترط استيعابها كلها مرة واحدة إلا أن فهمها قد يساعد
في حل بعض مشكلات الترجمة وتحسين جودة التَّوطين.

\subsection{الترتيب}
أحيانا ينبغي تغيير ترتيب المتغيرات عما في المصدر للوصول إلى أنسب صياغة
للترجمة. يَسهُل ترتيب المتغيرات ذات الأسماء الدّالة، لكن المتغيرات
الأكثر شيوعا ذات أسماء رمزية مثل “\Type{%d}” و “\Type{%s}”

\example{There are \%d photos on the device \%s}

وسيُظهر هذا شيئا مثل “There are 55 photos on the device Nikon XYZ”. لكن
قد يكون من الأنسب صياغة الترجمة على “يحتوي الجهاز Nikon XYZ على 55
صورة”، وساعتها ستكون عبارة الترجمة كتالي:

\example{يحتوي الجهاز \ltr{{\bf \%2\$s}} على \ltr{{\bf \%1\$d}} صورة}

هنا غيّرنا الأسماء الرمزية للمتغيرات ليدل الاسم على موضع المتغير في
العبارة المصدر بالإنگليزية. فالمتغير “\Type{%d}” يصبح “\Type{%1$d}”
بإدخال “\type{$1}” فيه، وهذا يعني أنه يقابل المتغير الأول في العبارة
الإنگليزية بغض النظر عن ترتيب وروده في الترجمة. قِس على هذا باقي المتغيرات.
تدعم كثير من البرمجيات هذا الأسلوب (خاصة المبنية بلغة سي و شبيهاتها التي توظِّف
دالة \Type{printf()})، يمكن التأكد بسؤال المبرمجين.

\subsection{الإغفال}
أحيانا نُضطَّر إلى ترك أحد المتغيرات بعدم استعماله في الترجمة، خاصة مع
بعض صيغ المعدود (مثل حالة الصفر في العربية التي تصاغ بالنفي) أو في حال
تحتّم أن تكون صياغة الترجمة مختلفة تماما عن المصدر. قد لا يكون هذا
ممكنا في كل البرمجيات، لذا عليك التأكد بسؤال المبرمجين، و تذكَّر أنك
بإغفال متغير فإنك قد تحجب معلومة عن المستخدم، و هو ما ينبغي تجنبّه إلا
للضرورة، أو أخذه في الحسبان.

في حال تعذّر إغفال متغير فربما أمكنت كتابته بطريقة تجعله لا يظهر عند عرض
النص، فيمكن كتابة المتغير “\Type{%s}” على الصورة “\Type{%.0s}” والتي ستمنع ظهور
قيمته.

\subsection[ref:43524426]{الوقت والتاريخ}
أحيانا تُستعمل المتغيرات لضبط
تنسيق الوقت والتاريخ. طالع قسم “تنسيق الأعداد والتواريخ” في صفحة
\at[ref:30596701] لمزيد من المعلومات. هذه الحالة قد تتطلّب
استعمال متغيرات غير ما في المصدر.

\subsubsection{مثال}
السطر التالي يُعيِّن تنسيق الوقت ليكون على نحو “8:15 م”:

\example{\%I:\%M \%p}

هذا نسق اثنتي عشرة ساعة مع إضافة “ص” و “م” إليها. في المَحليَّات التي
يفضل فيها نسق أربع وعشرين ساعة ستكون عبارة الترجمة:

\example{\%H:\%M}

وسيُظهر هذا الوقت ذاته على نحو “20:15” دون “ص” أو “م”. في حالة تنسيق
الوقت والتاريخ يمكن استعمال متغيرات مختلفة أو حتى عدد مختلف من
المتغيرات ولا مشكلة في هذا.

تختلف تفاصيل عمل هذه المتغيرات من لغة برمجة إلى أخرى، وإن كانت في الأغلب
متماثلة. هذه وثائق بعض اللغات البرمجة الشائعة في مشروعات البرمجيات
الحرة:

\startitemize[1]
\item {\bf سي و سي++}
\CURL{http://opengroup.org/onlinepubs/007908799/xsh/strftime.html}
\item {\bf بي.إتش.بي}
\CURL{http://php.net/manual/en/function.strftime.php}  أو 
\CURL{http://www.php.net/date}
\item {\bf پيثون}
\CURL{http://docs.python.org/library/datetime.html#strftime-strptime-behavior}
\item {\bf دوت نِت}
\CURL{http://msdn.microsoft.com/en-us/library/system.globalization.datetimeformatinfo.aspx}
\stopitemize

في بعض النُظُم تُمكن مراجعة دليل \type{strftime} وهو مماثل للرابط الأول أعلاه.

\subsection{التنسيق}
يمكن في بعض الحالات تحديد كيفية تنسيق قِيَم المتغيرات، وقد يفيد هذا في
تحديد التنسيق المطلوب للأرقام في برمجيات المحاسبة على سبيل المثال.
يمكنك التحكم في ظهور علامة الموجب والسالب (+\letterbackslash -) أو وضع
أصفار قبل العدد إلخ.

لا يطرأ هذا كثيرا وقد لا تعرض له أبدا. لمزيد من المعلومات، طالع:
\CURL{http://opengroup.org/onlinepubs/007908799/xsh/fprintf.html}

\chapter[ref:30364807]{أدوات التوطين}
تناولنا أداتي التوطين الرئيسيتين
في هذا الكتاب في صفحة \at[ref:20165030]. يكمل فِِرتال وبُوتِل
بعضهما البعض لتحقيق وظائف عديدة مما يحتاجه فريق الترجمة. يُمكِّن بُوتِل
من الترجمة عبر الإنترنت أو بدونها مع التركيز على المشاركات التطوعية،
بينما فرتال أداة ترجمة تركز على البساطة وتعظيم الإنتاجية وتعمل في وجود
اتصال بالإنترنت أو بدونه. بالإمكان تنزيل الملفات من بوتل لترجمتها في
فرتال على حاسوبك إن كنت تفضل هذا.


\placefigure [right,hang]
  {none}
  {\externalfigure[img003.png][width=2.5in]}

\section{فِرتال}
فرتال أداة ترجمة قوية وبسيطة في آن. يمكنك بها زيادة إنتاجيتك بمعرفة
اختصارات وحِيَل أخرى دون أن تشتتك واجهة متخمة بالوظائف.

مع أن أغلب وظائف فرتال يمكن الوصول إليها بالفأرة إلا أنها صممت لتساعد
على أداء العمل بلوحة المفاتيح قدر الإمكان لزيادة السرعة وجعل الترجمة
أكثر متعة.

\subsection{فتح ملف}
في أغلب الأحوال يمكن فتح ملف الترجمة بالنقر عليه في مدير الملفات، لكن في
حال كان الملف مرتبطا ببرمجية أخرى فيمكن البحث عن فرتال في قائمة السياق
بالنقر الأيمن على الملف.

يمكن أيضا تشغيل فرتال ثم فتح ملف من قائمة ملف ← افتح أو \KBD{Ctrl+O}.

تشمل أنساق الملفات المدعومة:

\startitemize[1]
\item Gettext PO
\item {\sc XLIFF} \rlm (\ltr{.xlf})
\item {\sc TMX}
\item {\sc TBX}
\item WordFast TM \rlm (\ltr{.txt})
\item مسارد OmegaT
\item Qt Linguist \rlm (\ltr{.ts})
\item Qt Phrase Book \rlm (\ltr{.qph})
\stopitemize

\subsection{أسلوب الترجمة المعتادة}
بعد فتح الملف ستظهر أول وحدة ترجمة و سيكون مؤشر الكتابة في الحقل أسفل
النص المَصدَر، فيمكن ببساطة كتابة الترجمة ثم ضغط \KBD{Enter} عند الانتهاء
من الوحدة، مثل أي محرر نصوص. لاحظ أن \KBD{Enter} تنقلك إلى الموضع التالي
الذي ستكتب فيه، ففي حالة وحدات ترجمة الجموع ستنقلك إلى السطر التالي في
نفس الوحدة.

إذا كان في النظام مدقق إملائي مُنصَّب فسيُفعَّل التدقيق الإملائي للترجمة
وللنص المصدر أيضا.

يمكنك التراجع عما كتبته بضغط \KBD{Ctrl+Z} كالمعتاد.

\subsection{أسلوب اختصار الوقت}
\subsubsection{الإكمال التلقائي}
يوفر فرتال عليك بعض الوقت بمحاولة إكمال الكلمات الطويلة. ستقترح وظيفةُ
الإكمال التلقائي كلمة محتملة ويمكنك قبول الاقتراح بضغط <Tab>، أو إذا لم
يكن الاقتراح هو المطلوب فيمكنك مواصلة الكتابة، وإذا قبلت اقتراحا لا
تريده فيمكنك التراجع عنه بضغط \KBD{Ctrl+Z} كالمعتاد.

\subsubsection{التصويب التلقائي}
يُوفر فرتال عليك بعض الوقت بتصويب بعض أخطاء الكتابة الشائعة، لكن هذا
التصويب يعتمد على اللغة فبعض اللغات لا يحتوي فرتال على معلومات تصويب
لها. ندعوك إلى المشاركة في المشروع لتحسين هذه الوظيفة للغتك.

إذا صوَّبَ فرتال تلقائيا شيئا لا تريده فيمكن التراجع عنه بضغط \KBD{Ctrl+Z}.

\subsubsection[ref:36503815]{نسخ النصّ المَصدَر إلى الترجمة}
في بعض الأحيان يكون من الأيسر نسخ
النص المصدر كما هو إلى الترجمة ثم استبدال الأجزاء التي تتطلب ترجمة،
خاصة في النصوص التي تحتوي على رَقْم XML أو كثيرا من المتغيرات التي
تستغرق كتابتها وقتا أطوَل مما إذا نسخت من النص المصدر. يمكنك نسخ النص
المصدر إلى حقل الترجمة بيسر بالضغط على \KBD{Alt+Down}.

في بعض اللغات يَستبدِل فِرتال علامات الترقيم تلقائيا لتناسب قواعد اللغة
الوجهة، مثل الفاصلة أو التنصيص أو غيرها من علامات الترقيم، أو المسافات
بين العناصر. مثلا \ltr{“item, item?”} ستتحول تلقائيا إلى “item، item؟” في
العربية دون تدخل من المترجم.

إذا كنت لا تريد هذه التغييرات التلقائية في النص فيمكن التراجع عنها بضغط
\KBD{Ctrl+Z}.

\subsubsection[ref:32596109]{نسخ مُدْرَج إلى الترجمة}
المُدْرَجات مقاطع خاصة من النص
يمكن إبرازها تلقائيا وإدراجها بسهولة في الترجمة. سترى أن بعض مقاطع النص
مُبرَزة على نحو خاص، ويمكن تحديدها بضغط \KBD{Alt+Right} ثم ضغط \KBD{Alt+Down}
لإدراجها في الترجمة، بعدها سيُبرَز المُدْرَج التالي.

\subsubsection{نسخ مصطلح إلى الترجمة}
تظهر النصوص التي يتعرف عليها فرتال مبرزة على نحو خاص، ويمكن التعامل معها
مثل المُدْرَجات، فيمكن استخدام \KBD{Alt+Right} و \KBD{Alt+Down} بذات الأسلوب.
إذا كان للمصطلح أكثر من اقتراح فسيعرض فرتال الاقتراحات في قائمة ويمكن
اختيار الترجمة المختارة أو ضغط <Escape> ومواصلة الكتابة.

\subsubsection[ref:36543315]{استعمال اقتراحات ذاكرة الترجمة أو الترجمة الآلية}


\placefigure[here,force]
  {none}
  {\externalfigure[img002.png][width=.8\textwidth]}
إن كانت لدى فرتال اقتراحات من ذاكرة الترجمة أو الترجمة الآلية فسيعرضها
تحت منطقة التحرير، و عندها يمكنك وضع المُقتَرَحِ الأول في حقل الترجمة
بضغط \KBD{Ctrl+1} أو الثاني بضغط \KBD{Ctrl+2} وهكذا، أو بالنقر المزدوج بالفأرة
على الاقتراح الذي تريده.

\subsection{التنقل}
رأينا سابقا كيف ننتقل إلى النقطة التالية في الترجمة بضغط \KBD{Enter}. يمكن
كذلك الانتقال بين الصفوف بضغط \KBD{Ctrl+Down} و \KBD{Ctrl+Up}، أو ضغط
\KBD{Ctrl+PgDown} و \KBD{Ctrl+PgUp} للانتقال بخطوات أكبر.

سينقلك فرتال بين الصفوف على نحو معين، في المعتاد ينقلك بين كل الصفوف،
لكن يمكنك اختيار طرق تنقل أخرى من أعلى نافذة فرتال.

\subsubsection{طَورُ النَّواقِصِ}
إن فعَّلتَ طور “النَّواقِص” فإن فرتال سيستعرض لك الوحدات التي لم تُترجَم
أو لم تُراجع، وهذا يُسرِّعُ لك إيجاد المواضع التي ينبغي العمل عليها.
ستظهر الترجمات في ترتيبها كما هو بين جاراتها في الملف لتطّلعَ على
السياق الذي تترجم فيه.

\subsubsection{طَورُ البحث}
عند تفعيل طَورِ البحث أو لو ضغطتَ \KBD{F3} سينقلك فرتال بين الصفوف التي تحوي
النص الذي تبحث عنه. ستظهر الترجمات في ترتيبها كما هو بين جاراتها في
الملف لتطّلع على السياق الذي تترجم فيه.

يمكنك الانتقال من حقل البحث إلى حقل الترجمة بضغط \KBD{Enter} أو باختيار
طَورٍ آخر.

\section{بُوتِل}
بوتل نظام لإنجاز الترجمات وإدارتها عبر الوب تستخدمه كثير من مشروعات
البرمجيات الحرة لإدارة صيرورة التوطين. يُعين بوتل على الترجمة عبر الوب،
وله خصائص عديدة تُيسِّر مراجعة الترجمة وتنظيم عمل الفريق، وهو مناسب
للعمل الجماعي وللمساهمين غير المتمرسين، ولمَرَثونات الترجمة.

يمكن لخادوم بوتل أن يستضيف مشروعا أو أكثر بأي عدد من اللغات. تُستخدم بعض
الخواديم لترجمة برمجيات مشروع واحد إلى لغات عدة، بينما تستخدم خواديم
أخرى لتنسيق عمل فريق لغة واحدة على مشروعات توطين برمجيات عدة. يتيح بوتل
إسناد أدوار مختلفة لأعضاء الفريق. يستطيع مستخدِمو بوتل الاستفادة من
كثير من وظائفه في تحسين جودة الترجمة، مثل مَسرَد المصطلحات واختبارات
الجودة.

ولأن بوتل نظام يعمل عبر الوب تقع مهمة تنصيبه وتحديد الوظائف المسموح بها
على عاتق مدراء الخواديم. إذا أردت ترجمة شيئ ما ليس موجودًا على الخادوم
فعليك الرجوع إلى مدير النظام. لا تنسَ أنه قد يكون لكل خادوم بوتل غرض
معين و طريقة مختلفة لإدارته ولإعطاء الصلاحيات للمتطوعين.

أغلب وظائف بوتل تتطلب ولوج المستخدم بحساب على الخادوم.

\subsection{إيجاد ما تترجمه}
سترى في صفحة بوتل الرئيسية المشروعات واللغات الموجودة على الخادوم، كما
ستجد ملخصًا بتقدم كل مشروع أو لغة يطلعك على درجة اكتماله ونشاطه. عند
اختيار لغة أو مشروع ستجد إحصائيات أكثر تفصيلا عن كل ملف في المشروع، كما
ستتاح لك وسائل الترجمة والمراجعة.

إذا أردت الترجمة سيتيح لك لِسانُ “ترجم” الوصول إلى خصائص الترجمة سواء
تنزيل الترجمات للعمل عليها على حاسوبك أو الترجمة في بُوتل عبر الإنترنت.
للوصول السريع إلى الأجزاء غير التامة من الترجمة اختر “ترجم بسرعة” للملف
أو الدليل الذي تريد. ويمكن تنزيل الملفات التي تريدها لترجمتها على
حاسوبك (باستخدام فِرتال مثلا) وبعد تمام الترجمة استخدم استمارة رفع
الملف التي في أدنى الصفحة لإيداع العمل في الخادوم.

لمراجعة الترجمة (بما فيها ترجماتك أنت) اذهب إلى لسان “راجع” الذي يتيح لك
الوصول إلى الاقتراحات التي أضافها أعضاء الفريق بالإضافة إلى اختبارات
الجودة التي قد تشير إلى مواضع تتطلب التحسين.

إذا كانت لك صلاحيات إدارة هذا المشروع أو اللغة فتستطيع أيضا ضبط
الصلاحيات وأداء مهام الإدارة الأخرى. راجع وثائق بُوتل لمزيد من
المعلومات عن هذه الخصائص.

بوسع مدراء الموقع إضافة لغات أو مشروعات جديدة وقوالب للترجمة (راجع قسم
“صيرورة توطين بسيطة” صفحة \at[ref:38565525] لمعرفة فائدة
القوالب). عادة ما تُنشر بيانات اتصال ليُمكن للمستخدمين التواصل مع مديري
الخادوم\letterbackslash مشروع الترجمة.

\subsection[ref:20355427]{المصطلحات}
قد يساعد بُوتل المترجمين في مسألة
المصطلحات، فيمكن وضع مسرد مصطلحات عام لكل لغة مع إمكانية تخطيها بإضافة
مسارد لكل مشروع على حدة. يمكن يحتوي مشروع يسمى “terminology” على أية
ملفات تستخدم لمطابقة المصطلحات.

يستطيع مديرو المشروعات توليد قائمة المصطلحات المتواترة من لسان “ترجم” في
واجهة بُوتل، ويمكن أن يساعد هذا في تسريع توحيد المصطلحات المتواترة، كما
يمكن إضافة مصطلحات أخرى إلى القائمة يدويًا.
\CURL{http://translate.sourceforge.net/wiki/pootle/terminology_matching}

\subsection{الاقتراحات}
يمكن في بوتل إعطاء المستخدمين صلاحية اقتراح ترجمات تحتاج من يراجعها قبل
اعتمادها في الترجمة النهائية، وتتحكم في هذا إعدادات صلاحيات الخادوم أو
المشروع.

يتيح هذا إسناد أدوار مختلفة لأعضاء الفريق مع إمكانية جعل المراجعة خطوة
إجبارية قبل اعتماد أي اقتراحات. كما يتيح هذا جمع ترجمات بديلة.
\CURL{http://translate.sourceforge.net/wiki/pootle/suggestions}

\subsection{اختبارات الجودة}
في بوتِل وظيفة لمراجعة جودة الترجمات بفحص عدة مؤشرات يمكنها أن تدل على
نقاط ضعف في الجودة. طالع قسم “المراجعة الآلية” في صفحة
\at[ref:36134309].

ليست كل المشاكل التي يشير إليها بُوتل أخطاء فعلية، إنما هو تنبيه باحتمال
وجود خطأ عليك مراجعته. على مدراء بوتل تحديد نوع المشروع (گنوم، كدي،
موزيلا، إلخ.) في صفحات الإدارة لمساعدة بوتل على إجراء اختبارات الجودة
المناسبة.
\CURL{http://translate.sourceforge.net/wiki/pootle/checks}

\subsection{البحث}
يوفر بوتل وظيفة بحث تتيح للمترجمين والمراجعين البحث في الترجمات. ستجد
مربع البحث في أعلى الصفحة، و يمكن استخدامه للبحث عن أشياء محددة تريد
العمل عليها أو لمطالعة كيف حُلَّت بعض المشاكل سابقا أو للتأكد من اتساق
الترجمات المختلفة.

نتائج البحث دوما حديثة وتعكس الترجمات الحالية الموجودة على بوتل.

\chapter{مشروعات التوطين}
\section[ref:38565525]{صيرورة توطين بسيطة}
في أبسط حالات الترجمة فإنّك تُعطى
ملفا واحدة لترجمته فتفتحه بأداة ترجمة مثل فِرتال ثم تعيده إلى المطورين
بعد تمام ترجمة العبارات الواردة فيه ومراجعتها. لكن عادة ما تكون الأمور
أكثر تعقيدا، مع أن بعض المشروعات تعمل بهذه الصيرورة، خاصة عندما تكون
المرة الأولى التي تُتَرجم فيها البرمجية إلى لغتك. يمكنك تبسيط الأمور
على نفسك بتفادي المشروعات المعقدة في البداية. في فصل دراسات الحالة
(صفحة \at[ref:34376426]) ستجد وصفا لمشروع “رسم تُكس” وهو
مشروع بسيط للغاية يصلح للبدء به.

في الحالات العامة قد يكون عليك إجراء خطوات إضافية أو قد يتوجب عليك توليد
ملفات الترجمة بنفسك متبعا خطوات محددة. في حال وجود ترجمة سابقة سيتعين
عليك تحديث ملف الترجمة ليحوي كل المقاطع الجديدة التي تُنتظر ترجمتها،
بحيث تتطابق الترجمة مع إصدارة البرمجية التي تترجمها.

تتبع كثير من المشروعات صيرورة مشابهة لما يلي:

  

\placefigure[here,force]
  {none}
  {\externalfigure[diagram][width=\textwidth]}

\startitemize[1]
\item يُوَلّد من الكود المصدري (ملفات البرمجة) قالب ترجمة يحوي كل النصوص
التي ستُترجَم.
\item تبدأ الترجمات الجديدة بترجمة هذا القالب، وملفات الترجمة القديمة
تُحدّث لتطابق القالب الجديد.
\item يُتَرجم الملف، وأحيانا يخضع كذلك للمراجعة.
\item تودّعُ النسخة النهائية في المشروع، وستُبنى عليها في المستقبل أي
ترجمات إلى هذه اللغة.
\stopitemize

لاحظ أنه لا يتوجب عليك أبدا البدء من الصفر في حال وجود أي ترجمات سابقة
بل، تبني عليها و تُحسِّنها.

\section{إيجاد الملفات المطلوبة}
لترجمة أي برمجية تحتاجُ الملفاتِ التي ستعمل عليها. كثير من المشروعات
توثق كيفية الحصول على ملفات الترجمة وإيداعها في المشروع عند إتمامها.
هذه الخطوات في الأغلب بسيطة للغاية في المشروعات التي تستخدم بُوتل، لكن
حتى المشروعات التي لا تستخدمه غالبا ما تحدد فيها خطوات واضحة.

كثير من المشروعات تستعمل ملفا واحدا لكل لغة وقد يكون كل ما عليك هو تنزيل
الملف ثم ترجمته وإرساله إلى المطور، وبعض المشروعات تضع تعليمات محددة
لإيداع الملف (مثل فتح بلاغ علة وإرفاق الترجمة به).

بعض المشروعات تتوقع منك أن تتعامل مع {\it نظام إدارة الإصدارات} الذي
يستخدمونه. تُستخدم نظم إدارة الإصدارات لإدارة كل ملفات المشروع بما فيها
ملفات الترجمة، وتُطوّر أغلب البرمجيات باستخدام هذه النظم، وفي حالة
مشروعات البرمجيات الحرة يكون محتواها متاحة للعموم. وقد يكون هذا هو
الأسلوب المفضل للحصول على الملفات ثم إيداعها بعد تمام العمل عليها.
تلزمك صلاحيات معينة لتستطيع إيداع الملفات في أنظمة إدارة الإصدارات وقد
يكون لكل مشروع نظام مختلف لمنح هذه الصلاحيات. في الغالب لست مضطرا إلى
فهم أنظمة إدارة الإصدارات لتوطين البرمجيات، لكن في بعض الحالات يكون
أفضل لك أن تعرف أساسيات عملها. يمكنك الاستعانة بالمطورين لإرشادك؛ و
إحالتك إلى أية وثائق تعنيك مطالعتها. يشرح قسم “نظم إدارة الإصدارات”
صفحة \at[ref:34384626] بعض أساسياتها.

سواء حصلت على الملفات بهذه الطريقة أو تلك فعليك التأكد من إن كانت
الملفات مُحدّثة أم أنّ عليك تحديثها بنفسك لتحوي آخر الإضافات من
المقاطع. وهذا أحد أسباب أنك يجب {\bf ألا} تبدأ الترجمة اعتمادا على أحد
إصدارات البرمجية المتداولة لأنها قد لا تحتوي على أحدث الملفات.

أغلب مشروعات البرمجيات الحرة لا تتطلب سوى ترجمة ملف واحد يكون غالبا في
نسق \Type{.po} لكن هناك أنساق أخرى مثل \Type{.ts} و \Type{.properties} و \Type{.php}،
وغالبا ما تكون الملفات في دليل يسمى “po” أو “locales” أو “lang” أو شيء
مماثل.

تعتمد كثير من المشروعات على برمجيات أخرى خارج المشروع الأساسي (مثل
المكتبات البرمجية) ولتكتمل ترجمة التطبيق تنبغي ترجمة تلك البرمجيات
الأخرى. اسأل المطورين إذا لم تكن متأكدا.

أحيانا يوجد إجراء معيّن لإضافة ترجمة مستحدثة، فمثلا تشيع الحاجة إلى
إضافة رمز اللغة في ملف يسمى LINGUAS، وقد يكون من المعتاد ذِكرُ إضافة
اللغة الجديدة في سِجِلِّ التغييرات (ملف يسمى ChangeLog). يتفاوت هذا من
مشروع لآخر وعليك مراجعة الوثائق.

في حال ما كان من غير الممكن فتح الملفات التي يوفرها المشروع مباشرة في
أدوات الترجمة فقد يتوجب في هذه الحالة تحويلها إلى أحد أنساق ملفات
الترجمة. في كثير من الحالات يكون هذا التحويل مهما و ذا مميزات عديدة.
اطلع على المزيد عن هذه المسألة في:
\CURL{http://translate.sourceforge.net/wiki/guide/monolingual}

تحتوي عُدَّة الترجمة على عدد من أدوات تحويل الصيغ التي تستخدمها كثير من
المشروعات الحرة (راجع صفحة \at[ref:20165030])، كما توجد
مُحوِّلات أخرى مثل po4a و xml2po\rlm.
\CURL{http://translate.sourceforge.net/wiki/toolkit/index}
\CURL{http://po4a.alioth.debian.org}


\section{التواصل مع المشروع}
عند انخراطك في أي مشروع ستحتاج للتواصل مع الآخرين. يشرح هذا القسم بعضا
من أساليب التواصل الشائعة في مشروعات البرمجيات الحرة.

\subsection{البريد الإلكتروني}
{\it القوائم البريدية} نظام لتوزيع رسائل البريد الإلكتروني إلى المشتركين
لتسهيل النقاشات الجماعية عبر البريد، فأي رسالة ترسل إلى عنوان القائمة
البريدية تُرسل تلقائيا إلى كل المشتركين في القائمة. تستخدم القوائم
البريدية للتواصل بين المشاركين في كثير من المشروعات، وقد توجد قائمة
واحدة للمطورين والمترجمين والمستخدمين، أو عدة قوائم بريدية؛ واحدة لكل
غرض. في الأغلب ستحتاج للاشتراك في أحد القوائم البريدية لتستطيع المساهمة
بفعالية في المشروع، وقد تكون هذه أفضل طريقة لتطلع على تطورات المشروع
والإصدارات الوشيكة والاستجابة إلى استفساراتك.

لاحظ أن كثيرا من مشروعات البرمجيات الحرة تفضل المحافظة على الغرض الأساسي
للقائمة البريدية، وأية نقاشات بعيدة عن غرض القائمة، أو دعاية، أو
استخدام للقائمة في الأمور الاجتماعية ليس مرحب به في الكثير من
المشروعات. كما قد توجد قواعد أخرى حسب أعراف كل قائمة، مثل السماح
بالمرفقات من عدمه إلى غير ذلك، و قد تستنتج بعضا من تلك القواعد من
الرسائل التي تصلك من القائمة، كما ينبغي عليك تجنب إضافة عنوان القائمة
البريدية إلى خدمات الشبكات الاجتماعية وما شابهها.

تذكر أن أغلب هذه القوائم البريدية مؤرشفة ومتاحة للعموم، وأي رسالة ترسلها
سيظهر بها على الأغلب عنوان بريدك وسيُتاح محتواها لأي شخص يستطيع الوصول
إلى الوب. التزم بآداب الحوار وابتعد عن الجدال غير المبرر وضع في حسبانك
تباين الخلفيات الثقافية لأعضاء القوائم وتفاوت مستويات تمكنهم من اللغة
الإنگليزية. حاول أن تكون مراسلاتك واضحة بقدر ما تستطيع.

لا يحبذ إرسال رسائل مباشرة إلى عضو بعينه في المشروع ما لم يكن هناك داع
(كأن تكون هذه الطريقة التي يعتمدها المشروع لإيداع الترجمات) أو أن تكون
بينكما معرفة شخصية. لذا دوما الرد على المراسلات على عنوان القائمة
البريدية، لضمان تضمين النقاشات في القائمة ليستفيد منها الجميع.

\subsection{قنوات الدردشة}
تتخذ العديد من المشروعات الحرة من خدمات الدردشة مثل IRC وسيلة إضافية
للتواصل، لمساعدة المستخدمين أو لنقاشات أكثر تفاعلية، وقد تكون هي
الوسيلة المثلى للحصول على إجابات سريعة أو الاستعلام عن مسألة تتطلب عدة
أسئلة وأجوبة. عادة ما تستخدم برمجيات الدردشة للانضمام إلى تلك القنوات
لكن يمكن أحيانا استخدام متصفح الوب.

قد يظهر أن أشخاصا عديدين موجودون في القناة لكنهم في الواقع قد لا يكونون
أمام حواسيبهم في ذلك الوقت وبالتالي لا يقرؤون رسالتك، لذا عليك أن تتحلى
بالصبر و المكوث في القناة ريثما تصلك الإجابة، فكثيرا ما يعود الأشخاص
إلى حواسيبهم ويردون عليك بعد ساعة مثلا. وقد يكونون في توقيت مختلف عن
توقيتك فلا يرون رسالتك إلا بعد استيقاظهم.

مثلما في القوائم البريدية، عليك الالتزام بآداب الحوار، كما قد يكون
سِجِلُّ المحادثات في القناة منشورا للكافة. إذا كان لديك سؤال فاسأله
مباشرة، فليس عليك الاستئذان قبل السؤال، ففي كثير من القنوات يفضل
المشاركين أن تصوغ سؤالك بكل التفاصيل المطلوبة منذ البداية.

لا تلصق أبدا مقاطع كبيرة من النصوص إلى القناة، فقد تقطع بعض الخواديم
اتصالك بسبب ذلك، كما أن سطور النص الطويلة الكثيرة تزعج المشاركين في
القناة. من الأفضل في هذه الحالات التراسل بالبريد الإلكتروني، أو وضع
النص الذي تريد إطلاع غيرك عليه على إحدى خدمات {\it pastebin}، و هي
خدمات على الوب يمكن لصق النص فيها الربط إليه من الحوار في القناة.

\subsection{مُقتَفِياتُ العِلّاتِ}
تَستخدم مشروعات كثيرة {\it مقتفيات علات} مثل بگزيلا أو تراك، وهي نظم
تُدرَجُ فيها المشكلات والطلبات والتحسينات المقترحة على البرمجيات
فتُناقش و يُخطط لها. عادة ما يوضع رابط إليها في موقع المشروع.

قد يُطلب منك إيداع الترجمات عبر أحد هذه النظم، كما قد تكون المكان
المناسب للإبلاغ عن أية أخطاء تجدها في النص المصدر الذي تترجمه. توجد
حُقول عديدة في استمارة البلاغ لكن لا تشغل بالك بها واترك ما لا تعرف
معناه منها على قِيَمِها المبدئية، ومع الوقت ستفهم الأسلوب المفضل في كل
مشروع لملء تلك الاستمارات.

\subsection{الويكيّات والمنتديات}
أحيانا يستخدم ويكي لبناء موقع المشروع أو وثائقه بما فيها توجيهات
المشاركة. قد تشاركُ في ترجمة موقع المشروع أو وثائقه وقد يجري ذلك عبر
الويكي، لذا عليك التآلف مع استخدام صيغ تنسيق النصوص في ويكي.

ليس من الشائع استخدام المنتديات في مشروعات البرمجيات الحرة، لكنها قد
تستخدم أحيانا لمساعدة المستخدمين أو للنقاشات بين المشاركين، وقد تُحبُّ
استخدامها لمساعدة المستخدمين بلغتك، و هي مساهمة قيّمة المشروع، لكن
استطلع إن كانت النقاشات بلغات غير الإنگليزية مسموح بها، أو اطلب قسما
للدعم التقني بلغتك.

إن كنت تؤمن بأهمية توطين البرمجيات فاعلم أن وجود المساعدة والوثائق بلغة
المستخدم مهمة أيضا. قد لا يتوفر عدد كاف من الناس لأداء كل هذه المهام
(خاصة للغات الصغيرة) لذا لا تنسَ تحديد أولوياتك ووضع أهدافك نصب عينيك.

\section[ref:34384626]{نظم إدارة الإصدارات}
عرجنا فيما سبق على نظم إدارة
الإصدارات. يوجد الكثير مما ينبغي تعلُّمُه عن هذه النظم لكنك في الأغلب
لا تلزمك سوى معرفة بعض الإجراءات الأساسية للمساهمة بالترجمة. اعرف أي
نظام لإدارة الإصدارات يستخدمُ المشروعُ وكيفية أداء المهام التالية به:

\startitemize[1]
\item جلب نسخة كاملة من المشروع بها كل الملفات
\item كيفية تحديث الملفات لاحقا إلى آخر التغييرات التي حدثت في المشروع
\item كيفية إيداع ملف
\stopitemize
تغطي هذه الأمور أغلب المهام المطلوبة منك. بالرغم من أنه يمكن التراجع عن
أي تغييرات باستخدام نظام إدارة الإصدارات إلا أن عليك الحرص ألا تُحْدِث
تغييرات بلا سبب. اعرف ما يلي:

\startitemize[1]
\item متى يُسمح لك بإيداع الملفات
\item إن كان إدخال رسالة في السجل مطلوبا عند الإيداع الملف
\item إن كان عليك التعديل في ملفات أخرى (مثل ملف LINGUAS)
\item إن كانت تجهيزات معينة مطلوبة قبل الإيداع مثل التحول إلى فرع آخر
\stopitemize
{\it الفروع} أسلوب للعمل بالتوازي على إصدارتين مختلفتين من برمجية ما في
الوقت ذاته. قد يتوجب عليك إيداع الترجمات في إصدارة بعينها وبالتالي في
فرع بعينه في نظام إدارة الإصدارات، وعليك سؤال المطورين للتأكد.

تشرح الصفحة التالية بعض مبادئ نظم إدارة الإصدارات:
\CURL{http://betterexplained.com/articles/a-visual-guide-to-version-control}

\section{جدول المشروع}
لكل مشروع برمجي جدوله و إيقاعه الذي يسير عليه. فالمشروعات التي تُصدِر
إصدارات جديدة بسرعة وتقبل الترجمات في أي وقت تمكنك المشاركة فيها في
الوقت الذي يناسبك. أما المشروعات التي تُصدِر في مناسبات معينة فتتطلب
تخطيطا جيدا لمشاركتك في المشروع لتضمن أن يستفيد من عملك أكبر عدد من
الناس.

من المُعتاد أن تُعلِنَ المشروعاتُ عن خُطط الإصدارات مقدما وأن تَطلبَ من
المترجمين تحديث ترجماتهم، و يكون هذا الوقت هو الأنسب لتساهم فيه
بترجمتك، إذ قد ينم هذا عن انصباب اهتمام المطور بالمترجمين وسعيه لتوفير
بيئة مناسبة لمساهماتهم و تقديم الدعم لهم. إذا كنت تستحدث ترجمة جديدة
فاحرص على أن يكون أمامك وقت كافٍ للانتهاء من الترجمة واختبارها. اقرأ
المزيد في قسم “العدُّ والتقدير” في صفحة \at[ref:33531520] و
قسم “الاختبار والمراجعة” في صفحة \at[ref:34441726].

إذا كنت تترجم مشروعا ضخما للمرة الأولى فقد يكون عليك البدء في العمل قبل
مثل هذا الإعلان بفترة كبيرة على أيّ حال، لذا فلن يؤثر الإعلان كثيرا
خلال القسم الأكبر من عملك.

يتواكب الإعلان عن تلقّي الترجمات مع مرحلة تسمى {\it تجميد النصوص} وتعني
تثبيت النصوص للإصدارة الوشيكة فلا تُحدث فيها أية تغييرات، وهو ما يتيح
للمترجمين فرصة العمل على الترجمة دون خشية تغير النص قُبيل الإصدار. وقد
يحدد هذا الإعلان الإصدارة الوشيكة و أي أفرع نظام إدارة الإصدارات
سيُستخدم. فإن لم تكن متأكدا فاسأل المطورين ولا تخاطر بالعمل على
الإصدارة الخطأ.

إذا بدأت متأخرا كثيرا فقد لا تتم العمل في الموعد فتتأخر على موعد قبول
المساهمات في تلك الإصدارة. البدء مبكرا آمن لكن أحيانا لا يكون أفضل،
فإذا بدأت قبل تجميد النصوص فقد تطرأ تغييرات قبل إعلان تجميد النصوص. لكن
إن استشعرت أن التغييرات المحتملة في النصوص قد تكون قليلة،و الوقت اللازم
لتداركها أقل من الوقت اللازم للعمل على الترجمة كلها، فقد يكون من المجدي
العمل على الترجمة على أية حال. متابعة أخبار المشروع تجنبك الإحباطات.

\section{البرمجيات الحرة}
\subsection{الدوافع}
كما ذكرنا سابقا، تختلف دوافع الناس للمشاركة في مشروعات البرمجيات الحرة،
البعض يتخذها هواية أو لحل مشكلة واجهتهم شخصيا، والبعض يعمل في شركات
تساهم في البرمجيات الحرة لدوافع عديدة. المهم أن تعي احتمال اختلاف دوافع
الناس الذين تقابلهم في البرمجيات الحرة عن دوافعك، وعندما تطلب المساعدة
من شخص ما فهذا يستغرق منه وقتا و جهدا وقد ينشغل بعمله عن مساعدتك، وقد
لا يهتم شخص ما بمساعدتك لأن الأمر الذي تطلب المساعدة فيه لا يستهويه.
عادة ما يسعى الناس لمساعدتك لكن عليك إدراك أن أغلبهم غير ملزمين بذلك،
لذا عليك التأدب دوما، و بالذات عند طلب المساعدة. كما عليك السعي لمساعدة
الآخرين بقدر استطاعتك. ثقافة عمل المعروف منتشرة في أوساط البرمجيات
الحرة، لذا تذكر أنك تطلب معروفا وأن عليك أن تُسدي مثله للآخرين.

\subsection{الرُّخص}
تُوَزَّعُ البرمجيات بشروط معينة تحدد ما المسموح لك به فيما يخص كل
برمجية. تؤلِّف هذه البنود الرخصة التي عادة ما تُعرَض عليك أثناء تنصيب
البرمجيات، أو ربما توضع في ملف ضمن الكود المَصدَريّ. توزع البرمجيات
الحرة مفتوحة المصدر بِرُخصَ جذابة للغاية تتيح للمستخدم فعل الكثير بها،
بما في ذلك توطين تلك البرمجيات. توجد بعض مسائل الرُّخص وحقوق النشر تهم
المترجمين.

أولا، اعلم أن مساهماتك في المشروع ستوزع بالرخصة ذاتها التي توزع بها باقي
البرمجية. من المستحسن أن تطّلع على بنود هذه الرخص لتعرف ما الذي يتاح
للآخرين فعله بترجمتك، فمثلا أغلب رخص البرمجيات الحرة تسمح ببيع نسخ من
البرمجية، لذا من الأفضل ألا تتفاجأ بشيء كهذا.

بعض المشروعات تشترط لقبول المساهمات توقيعَ المساهم على {\it استمارة نقلٍ
لحقوق النشر}. تختلف الدوافع وراء هذا، فقد تكون لإثبات هويتك الحقيقية أو
لإعطائهم الحق في تغيير الرخصة المنشورة بها البرمجية لاحقا، أو لإعطائهم
سندا قانونيا لتمثيلك أمام المحاكم حال وقوع نزاع مع آخرين بشأن البرمجية.
وهو أمر غير شائع لكنك قد تقابله في بعض المشروعات.

في بعض الأحيان لا يُسمح لك بتوزيع نسخة من البرمجية ترجمتها بنفسك خارج
إطار المشروع بذات العلامات التجارية (مثل الاسم والشعار) للمشروع الأصلي.
استفسر عن أية اشتراطات تخص العلامة التجارية للتيقن.

ضع في اعتبارك أن للمشروعات الحرية في استعمال ترجمتك وفقما تحدده الرخصة
وأنك لا تستطيع سحب الحقوق التي منحتها للمشروع بإيداعك للترجمة. بل قد
يُسنِد المشروعُ مهمة الترجمة إلى شخص غيرك فيغير بعضا من الترجمة دون إذن
منك، بالرغم من ندرة حدوث هذا إلا أنه يكون مطلوبا أحيانا للحفاظ على
استمرارية المشروع، وينبغي ألا يؤثر هذا على عملك.

لمزيد من المطالعة طالع “ترخيص البرمجيات الحرة\letterbackslash مفتوحة
المصدر” من تأليف Shun-ling Chen\rlm.
\CURL{http://www.iosn.net/licensing/foss-licensing-primer/foss-licensing-final.pdf}
\CURL{http://en.wikibooks.org/wiki/FOSS_Licensing}

\subsection{مشاركة أوسع}
من الجوانب الرائعة في مشروعات البرمجيات الحرة ترحيبها بالمساهمات في كل
الأنشطة المتعلقة بالمشروع سواء الترجمة أو التوثيق أو الاختبار أو
التسويق أو البرمجة أو التصميم؛ كلها مساهمات مفيدة. قد تكون مهتما
بالتوطين فقط لكنك قد تكتشف أن المشاركة في جوانب أخرى مفيدة، ليس للمشروع
فحسب بل أيضا لعملك على التوطين.

{\bf مراجعة النص الإنگليزي} تفيد مستخدمي الواجهة الإنگليزية وفي الوقت
ذاته تعين المترجمين في عملهم. إذا لم تكن تتقن الإنگليزية فإن قيامك
بالتنويه عن المواضع في النص التي تصعب ترجمتها أو تستحيل يفيد في تحسين
النص المَصدَر وهو ما يعود بالتحسين على عملك في الترجمة. فالمترجم من
القلائل الذين يقرؤون النصَّ المَصدَر في الإنجليزية بتمعن بالغ مما يجعله
مراجعا جيدا. تستطيع في الأغلب الإبلاغ عن الأخطاء في النص المصدر
باستخدام مقتفي علّات المشروع.

{\bf اختبار البرمجيات} في واجهتها المترجمة خطوة مهمة في عملك على
توطينها. نتناول موضوع الاختبار لاحقا في صفحة
\at[ref:34441726]. من الفوائد الثانوية لهذا إمكان مساهمتك في
اختبار البرمجية ذاتها قُبيل الإصدار، كما أنك أفضل من يختبر التوافقية مع
أي من الأمور المتعلقة بلغتك أو بلدك مثل التعامل مع النصوص المكتوبة
بلغتك أو التوافق مع مواقع الوِب أو المحلية أو دعم الخطوط إلخ.

{\bf تصميم البرمجيات} لتناسب مجموعات متباينة من المستخدمين حول العالم
ليست مهمة سهلة وهذا أحد الجوانب التي تستطيع مساعدة المشروع بها للتأكد
من أن التصميمات تناسب لغتك وثقافتك. قد تكون مجرد أمور بسيطة مثل طلب
مساحة أكبر لمقطع طويل في الترجمة، وقد تكون أمورا أعمق في التصميم مثل
إضافة خصائص تحتاجها لغتك لم ينتبه لها المطورون.

{\bf تسويق البرمجيات} عبر العالم قد يكون مجهودا كبيرا على مجموعة صغيرة
من المتطوعين إلا إذا كان مساهمو المشروع من بلدان كثيرة. يمكنك أن نكون
سفيرا للمشروع في بلدك أو في إقليمك وقد تكون الترجمة جزءا من هذا أيضا؛
مثل ترجمة المواد التسويقية ومواقع الوِب والمطويات الترويجية وغيرها، وقد
يعتمد المشروع عليك للنجاح في المنطقة التي تعيش فيها.

{\bf المشاركة في الفعاليات الاجتماعية} ليست من الأنشطة المعتادة لكثيرين
من المساهمين في مشروعات البرمجيات الحرة، لكن قد تكون أمامك فرصة لتمثيل
مشروع تعرف الكثير عنه فعالية محلية، وهذه طريقة رائعة للترويج للمشروع
وربما دعوة مساهمين جدد لمساعدتك في التوطين.

\chapter{ما بعد الترجمة}
تناولنا في الأقسام السابقة بعض الجوانب المهمة في توطين البرمجيات الحرة،
ومع الوقت وبالممارسة ستصبح الأمور أسهل وستعتادها. الأمور التي نتناولها
في هذا القسم يمكن أن تحسن عملك أكثر وأكثر، وستساعدك في تنظيم وقتك على
نحو أفضل.

\section[ref:34441726]{الاختبار والمراجعة}
اختبار الترجمة أمر في غاية
الأهمية، حدوث الأخطاء أمر طبيعي لكن كثرة الأخطاء في الترجمة تعطي
المستخدمين انطباعا سيئا وتحرمهم من الاستفادة من البرمجيات المترجمة.
يشرح هذا القسم بعضا من أساليب الاختبار.

\subsection[ref:36134309]{المراجعة الآلية}
يُقصد بالمراجعة الآلية في هذا
الكتاب استخدام البرمجيات لأتمتة مراجعة الترجمات. قدرة البرمجيات على
إتقان المهام البسيطة المتكررة تؤهلها لاكتشاف الأخطاء الإملائية
والمسافات الزائدة بالإضافة إلى أخطاء المُتَغيِّرات والرَّقْم. كثير من
الأمور التي ناقشنها في قسم “مسائل تقنية” صفحة
\at[ref:32352020] يمكن أن تكتشفها أدوات المراجعة الآلية.

يحتوي كلا من بُوتِل وفِرتال على عديد من فحوص الجودة التي تساعد في اكتشاف
أنواع معينة من الأخطاء، بعضها متعلق بالتنسيق (مثل المسافات وعلامات
الترقيم) وبعضها أخطر (مثل نسيان مسار وِب أو ترجمةٍ أقصرَ من اللازم)
وبعض الفحوص قد تشير إلى أخطاء فادحة قد تؤدي إلى انهيار البرمجية
المترجمة أو تعطيل وظائفها (مثل أخطاء المتغيرات).

طالع المزيد عن فحوص الجودة هنا:
\CURL{http://translate.sourceforge.net/wiki/toolkit/pofilter_tests}

من المهم إدراك أن بعض هذه الاختبارات قد تنوِّه عما تحسبه أخطاء في حين
أنها قد لا تكون أخطاء فعلية، لذا لا تتبع توصيات هذه الفحوص بلا تدقيق
ولا تجعل نتائجها تدفعك إلى تعديل الترجمة على نحو لا يناسب لغتك لمجرد
إخماد الإنذار، فليست هذه الفحوص إلا أدوات لتعينك، و أنت أدرى بلغنك، وقد
وجدنا هذه الفحوص مهمة للغاية في عملنا خاصة للمترجمين الجدد، مع أن حتى
أكثر المترجمين خبرة يقعون في الأخطاء.

الفحوص الآلية لا تستغرق طويلا من وقتك، و هي تبين لك الأجزاء التي تتطلب
عناية أكثر لعلاج مشكلة محتملة. لذا راجع دوما نتائج فحوص الجودة فقد
تجنبك أخطاء.

البرمجيات التي تترجم بنظام جتتكست تستخدم عادة أداة اسمها {\bf msgfmt}
لتحويل ملف الترجمة إلى الملف التي يمكن للبرمجية استخدامه، كما تجري هذه
الأداة بعض الفحوص (مثل فحص تنسيق الملف و الاستعمال الخاطئ للمتغيرات) و
غيرها،و الأخطاء التي تكتشفها هذه الأداة {\bf يجب} تصويبها فغالبا لن
تُستعمل الترجمة إن وَجَدَت msgfmt فيها أخطاء. البرمجيات التي تعتمد على
نظام Qt للترجمة تستخدم أداة أخرى اسمها {\bf lrelease}.

\subsection[ref:37415529]{المراجعة اليدوية}
مع أن المراجعات الآلية يمكن أن
تساعد كثيرا إلا أنها لا تُغني عن مراجعة مترجم آخر. عندما تنضم إلى فريق
ترجمة فغالبا ستجد أن منسق الفريق يراجع ترجمات باقي الأعضاء، ويمكن أن
تطلب من شخص غيرك مراجعة ترجمتك وتنبيهك إلى المواضع التي قد تحتاج
للتحسين، فقراءة شخص آخر للترجمة كثيرا ما تساعد على تحسين جودتها، وأنسب
أداة ترجمة لهذا العمل التي تدعم التدقيق الإملائي وفحوص الجودة.

عادة ما يكون لفرق الترجمة الناضجة نظام يحدد دورة العمل بين المترجمين
والمراجعين، وأحيانا قواعد تحتم مراجعة كل الملفات. توفر بعض المشروعات
أدوات للمساعدة في تطبيق دورات عمل كهذه والتواصل بين أعضاء الفريق، وموقع
توطين مشروع گنوم من الأمثلة الرائعة على هذا.

عندما تعمل على ملفات PO فمن السهل وضع علامة “مبهم” على الترجمات التي ترى
أنها تحتاج للمراجعة، ويستطيع المراجع إزالة العلامة بعد مراجعته للترجمة.
قد يكون هذا معيقا أحيانا إلا أنها أبسط طريقة لإدخال المراجعة اليدوية في
صيرورة الترجمة.

على المراجعين الانتباه لقواعد الإملاء والنحو وغيرها من القواعد اللغوية
بالإضافة إلى الاتساق بين الترجمات وتوحيد استعمال المصطلحات وغيرها من
الأمور المهمة الأخرى. وعندما تراجع ترجمات الآخرين حاول أن تكون تعليقاتك
على الترجمة مهذبة لتشجعهم على الاستمرار.

\subsection[ref:34365426]{المراجعة العملية}
من الخطوات المهمة في صيرورة
التوطين مراجعة الترجمة بصريا أثناء تشغيل البرمجية. يتيح هذا لك أن ترى
نتيجة عملك، وخاصة متعة رؤية البرمجية مترجمةً للمرة الأولى. أساسيات
المراجعة واحدة بغض النظر عن طريقة بناء البرمجية أو تشغيلها، لكن قد
تختلف أساليبها من برمجية لأخرى، لذا سنعرض للحالات الأشهر في القسم
التالي.

الغرض من المراجعة العملية مراجعة الترجمة كما سيراها المستخدمون، وبهذا
يتضح أن الأمر يتعدى مجرد الترجمة إلى كل ما سيتعرض له المستخدم أثناء
تفاعله مع البرمجية. لاختبار الترجمة كما ينبغي عليك محاولة مطالعة كل
أجزاءها، تنقَّل نظاميا بين القوائم واحدة تلو الأخرى وانظر في كل خيار
وكل مربع حوار، انظر في التلميحات التي تظهر عندما تستبقي مؤشر الفأرة فوق
بعض عناصر واجهة البرمجية، بل وعليك محاولة فعل أشياء غير معتادة مثل فتح
ملفات لا تدعمها البرمجية، و إدخال قيم غير مقبولة في تضبيطاتها وملاحظة
رسائل الأعطال التي تظهر؛ فحتى رسائل الأعطال تُترجَم وتُراجَع.

من الأمور التي ينبغي الالتفات إليها مراجعة اتساق الأسلوب و توحيد
المصطلحات، وانسياب العبارات في السياق الذي ترد فيه على الشاشة، وكون
نصوص الشرح في الواجهة التي تشير إلى بعض تحكّمات البرمجية تستعمل نفس
العناوين المكتوبة على تلك التحكّمات. لا تنس أن البرمجيات كثيرا ما تتضمن
ترجمات من ملفات عدة قد يكون عمل عليها أشخاص عديدون لذا في الاتساق مسألة
مركبة.

من المهم كذلك ملاحظة أي انحراف في مظهر البرمجية بسبب الطول أو القصر
المفرطين للترجمة، ففي بعض الأحيان لا يضع المبرمج في حسبانه أن نصا قد
يكون بالطول أو القصر الذي عليه ترجمتك وبالتالي تفقد الواجهة تناسقها.
أغلب البرمجيات الحرة تتكيف بسلاسة مع طول الترجمة، لكن قد يتطلّب الأمر
ضبط أبعاد عناصر الواجهة بنفسك في قليل من المشروعات. يستغرق هذا الأمر
مزيدا من الوقت في الاختبارات، كما قد يلزم اختبار البرمجية على أنظمة
تشغيل مختلفة.

المُسرِّعات من الأمور التي ينبغي ملاحظتها أثناء تشغيل البرمجية. تناولنا
المسرعات سابقا في صفحة \at[ref:34484726]، والآن علينا التأكد
من أننا عيّنّا المسرعات المناسبة، وتصويب أي أخطاء نجدها. من المهم
اختبار كل مسرع في كل القوائم وكل مربعات الحوار. تحتاج أحيانا لضغط \KBD{Alt}
لتظهر السطور تحت المسرعات. تأكد من أن المسرع يُفعّل العنصر الصحيح، و
لاحظ ما يحدث عندما تضغط المسرع ذاته مرتين، فإن كان المسرّع ذاته مُعيّنا
لأكثر من عنصر فإن العناصر تُفعَّل بالتبادل مع كل ضغطة، و هذا في العادة
ليس مشكلة حرجة، لكن يُفضّل فضَّ هذا التداخل إن أمكن. قد يلزم التخطيط
بعناية لأي المسرعات تُعيِّن، لكن تذكر أن تُبدي أولوية للوظائف الحيوية
خاصة تلك الموجودة في نافذة البرمجية الرئيسية. كذلك حافظ على اتِّساق
مسرعات العناصر الشائعة (مثل “ملف” و “مساعدة”) بين التطبيقات المختلفة.

إذا كنت ترجمت الوثائق فعليك مراجعتها في عارض وثائق المساعدة المناسب.
تحقق من صحة عناوين الصفحات و من مسارات الروابط ما بين الصفحات، وأن
وظيفة البحث تُظهر نتائج مترجمة. تأكد من أن نصوص المساعدة تستعمل فيها
المصطلحات ذاتها التي في واجهة البرمجية. مراجعة متن المحتوى لا تختلف عن
مراجعة أي وثيقة مترجمة.

\subsection{اختبار البرمجيات التي تستخدم جتتكست}
تستخدم كثير من مشروعات البرمجيات الحرة حزمة برمجية تسمى جتتكست (Gettext)
لتطبيق دعم تعدد اللغات في المشروع. في مثل هذه المشروعات تترجم ملف PO
ويُوَلَّد منه ملف MO للاختبار، بعض أدوات الترجمة تحدث هذا بيسر، وإلا
فعليك استعمال أداة {\bf msgfmt} من سطر الأوامر:

\example{msgfmt  -cv  current\_translation.po  -o program\_name.mo}

ستجري هذه الأداة بعض الفحوص الأساسية (انظر صفحة
\at[ref:37415529]) وستبلغك بأي مشاكل خطرة واجبة التصويب، فإذا
لم تجد مشاكل خطرة ولَّدت ملف باسم مثل “program\_name.mo”، والاسم الفعلي
لهذا الملف يتوقف على الملف المصدر الذي مررته إليها، لكنه عادة ما يطابق
اسم البرمجية مكتوبا بأحرف لاتينية صغيرة متبوعا بالامتداد “\Type{.mo}”.

الآن انسخ ملف \Type{.mo} الذي ولَّدته إلى موضع وجود ملفات الترجمة (بعد أخذ
نسخة احتياطية من الملف السابق إن وجد)، فيما يلي أكثر أسماء الأدلة
المستخدمة شيوعا (استبدل ‘xx’ برمز لغتك):

\startitemize[1]
\item \ltr{\tt /usr/share/locale/{\it xx}/LC\_MESSAGES/}
          {\bf (في أغلب نظم لينُكس)}
\item \ltr{\tt C:\letterbackslash Program Files\letterbackslash {\it Program
Name}\letterbackslash share\letterbackslash locale\letterbackslash
{\it xx}\letterbackslash LC\_MESSAGES\letterbackslash}
          {\bf (في وِندوز)}
\stopitemize
قد يتوجب إنشاء هذه الأدلة إن لم تكن موجودة (خاصة في وندوز) ولا تنسَ ضبط
لغة النظام إلى اللغة المطلوبة.

\section[ref:36383525]{تحديد الأولويات (عندما لا تستطيع فعل كل شيء)}
بقدر ما نرغب في ترجمة كل البرمجيات
إلى لغتنا إلا أننا دوما تخضع لقيود الوقت وقلة المساهمين، وزيادة العمل
المطلوب يوما بعد يوم. هل نتخلى عن التوطين إذا؟ بالطبع لا، فقد رأينا
مِرات عديدة في عالم البرمجيات الحرة أن فريقا صغيرا أو حتى شخصا واحدا
يقدر على إنجاز الكثير وتوطين قدر كبير من البرمجيات. يستعرض هذا القسم
بعضا من الأفكار التي تساعدك تعظيم أثر عملك، و تعطي فكرة عما يمكن فعله
عندما لا تستطيع كل شيء. لا تنس أن تضع أهدافك نصب عينيك.

\subsection[ref:33531520]{العدُّ والتقدير}
إذا كان العمل كبيرا، أو كنت تظن
هذا، فمن المفيد للغاية معرفة مقدار العمل المطلوب لترجمة ملف أو برمجية
ما. أغلب مشروعات البرمجيات الحرة تُحصي عدد المقاطع (الرسائل) لكن هذا لا
يرسم صورة واضحة عن مقدار العمل المطلوب، عدد الكلمات معيار أدق لحساب
العمل الذي تستغرقه الترجمة.

بعض المشروعات تُبيِّن ملخصا بتقدم كل لغة. انظر هل يَعُدُّون الكلمات أم
المقاطع. أغلب أدوات الترجمة يمكن أن تعطيك ملخصات كهذه أيضا، ينحو بُوتِل
وفِرتال إلى ذكر عدد الكلمات لا عدد الرسائل، كما يمكن عدّ كلمات أي عدد
من الملفات تريد من سطر الأوامر بيسر باستخدام أداة “pocount” من عُدّة
الترجمة (انظر صفحة \at[ref:20165030]). 

لا يبين هذا طبعا درجة صعوبة الترجمة أو بأي قدر ستفيدك ذاكرة الترجمة أو
حتى جودة النص المَصدَر، و هذه و غيرها عوامل تحدد الوقت الذي تستغرقه
الترجمة، ومع هذا ستفيدك هذه المعلومات في التخطيط لعملك.

\subsection[ref:36346325]{اختيار منتَج أو مشروع}
قد تعيِّن لك أهدافُك أي البرمجيات
تريد أن تترجم، إما إن كنت لم تقرر بعد فعليك أن تحاولَ جمع بعض المعلومات
عن كل مشروع لتساعدك في اتخاذ قرارك:

\startitemize[1]
\item ما مقدار العمل المطلوب؟
\item ما مدى سهولة إيداع ترجمات جديدة؟
\item هل تدعم البرمجية خصائص لغتك، مثل الكتابة من اليمين لليسار أو
الخطوط التي تتطلبها؟
\item هل ستستطيع استخدام أدوات الترجمة التي تفضلها؟
\item ما شريحة المستخدمين الموجه إليها؟ قد لا يكون للبرمجيات المتخصصة
الأثر الذي تصبو إليه.
\item هل يعمل على أنظمة تشغيل مختلفة مثل وندوز ولينُكس؟ هل تتوفر إصدارات
لكل الأنظمة المدعومة في الآن ذاته؟
\item هل سيستطيع المستخدمون تفعيل الواجهة الموطنَّة بيسر أم يتطلب الأمر
تضبيطات؟
\item هل يتطلب توطين برمجيات أخرى مثل المكتبات البرمجية؟ وإن كانت
الإجابة نعم فيتوجب بحث إجابة كل هذه الأسئلة عن تلك البرمجيات الإضافية
أيضا.
\stopitemize
بالطبع لست ملزما بقصر مساهمتك على المشروعات المثالية من كل هذه الجوانب؛
يمكن أن تترجم مشروعا ما لأن الأمر ممتع أو لأنك تستخدمه، لكن المشروعات
الصعبة ستقلل من سرعتك وقد يكون الأمر محبطا لك.

\subsection{التوطين الجزئي}
في أي برمجية نترجمها نرغب عادة في ترجمة كل شيء، لكن بعض الأجزاء أهم من
غيرها، فإذا كنا لن نستطيع ترجمة كل شيء فمن الأفضل أن ترجم الأجزاء الأهم
أولا، خاصة إذا كانت هذه أول ترجمة بلغتنا. لذا علينا البحث عن وسائل
لتعظيم مخرجات العمل مع تقليل نطاق العمل.

في بعض الحالات، خاصة في النصوص ذات الطبيعة التقنية المتخصصة، قد يكون من
الأفضل ألا تُترجَم بعض الأشياء. إن كان رسائل الأعطال وما شابهها تحتوي
على مصطلحات جديدة عديدة لا ترجمات راسخة لها فقد ينفرُ المستخدمون
المعتادون على المصطلحات الإنگليزية من الترجمة. تذكَّر أن جمهورك يتطور
معك؛ فكلما زاد استخدام الناس للبرمجيات الموطنة كلما اعتادوها أكثر. إن
كنت تستحدث ترجمة جديدة فربما كان من المستحين تجاوز رسائل الأعطال
التقنية التي تتعلق بدواخل البرمجية والتركيز على الأجزاء التي سيراها
المستخدمون أكثر، فالأجزاء الظاهرة أكثر للمستخدمين هي ما سيبنون عليه
انطباعهم.

من الوسائل السهلة لتقليل عدد الكلمات أن تترجم ملفات محددة إذا كان
المشروع يتألف من ملفات كثيرة، فإذا أمكن تحديد الملفات الأهم بسهولة
فسيساعدك هذا في خطواتك الأولى. أما إن كان المشروع من ملف واحد أو كانت
الأجزاء المهمة موزعة على كل الملفات فيمكن استخلاص الأجزاء المهمة. أداة
“pogrep” من عُدّة الأدوات يمكن أن تكون عونا كبيرا لك في هذه المسألة
(انظر صفحة \at[ref:20165030]). يمكنك مثلا استخلاص المقاطع
القصيرة و حدها، أو تلك التي لا تأتي من ملفات برمجية يحتوي اسمها على
كلمة “admin” (وظائف الإدارة). طالع المزيد من المعلومات عن هذه الأداة
هنا:
\CURL{http://translate.sourceforge.net/wiki/toolkit/pogrep}

\section{توطين ما هو خلاف الواجهة و غير النصوص}
رغم أننا تناولنا أمورا كثيرة متعلقة بالترجمة إلّا بعض المشروعات لا تقتصر
على الترجمة وحدها؛ فبرمجية لتعليم الكتابة على لوحة المفاتيح تحتاج
لمعلومات عن أحرف لغتك وكلمات للتمارين مرتبة حسب درجة صعوبتها، و معالج
النصوص قد يحتاج إلى قائمة بالأخطاء الشائعة التي يمكن تصويبها، و بعض
اللغات يمكن تضمينها ملفات صوتية تنطق بلغتك، إلخ.

أثناء الترجمة قد تجد أخطاء في النص المصدر أو تكتشف صعوبة ترجمته على نحو
واضح لأن المبرمج أو كاتب النصوص الأصلية لم يضع في اعتباره الاختلافات
بين اللغات والثقافات المتعددة. ساعد المبرمجين بإبلاغهم عن هذه الحالات
فهذا يسهِّلُ الترجمة على الآخرين.

هذه المواضيع خارجة عن نطاق هذا الكتاب لكننا نتناول في هذا القسم بعض
المسائل الشائعة في توطين ما هو غير النصوص.

\subsection[ref:30596701]{تنسيق الأعداد والتواريخ}
تختلف أساليب كتابة الأعداد
والتواريخ من بلد لآخر، فبعض اللغات تستعمل في كتابتها أرقام تختلف عن
الأرقام الشائعة، وحتى اللغات التي تستعمل الأرقام الشائعة قد تختلف فيها
طرق كتابة الكسور أو تمييز الأعداد الكبيرة؛ فالعدد والتاريخ ذاتهما
يكتبان بصور مختلفة حول العالم.

\subsubsection{مثال على الأعداد}
\rexample{10509 (غير منسق؛ عشرة آلاف وخمسمئة وتسع)}

\rexample{10,509 (الولايات المتحدة)}

\rexample{10.509 (ألمانيا)}

\example{{\mal ൧൦،൫൦൯} (الهند؛ مالايلام)}

\subsubsection{مثال على التاريخ}
\rexample{15 مارس 2010 (2010-03-15)}

\example{March 15، 2010، 03/15/2010 (الولايات المتحدة)}

\example{15. März 2010 (ألمانيا)}

انتبه إلى النسيق الصحيح للأعداد والتواريخ في لغتك. بعض المشروعات تطلب
منك تحديد كيفية تنسيق الأرقام، عادة ما يكون هذا جزءا من المَحلَّية
(انظر صفحة \at[ref:34561726]) لكنه أحيانا يتكرر وروده في بعض
البرمجيات بل وقد تُستعمل له متغيرات خاصة في ملف الترجمة، راجع قسم
متغيرات الوقت والتاريخ صفحة \at[ref:43524426].

\subsection{العملة ووحدات القياس}
عندما تعرض البرمجيات قيما مالية (في برمجيات المحاسبة مثلا) فإنها لا
تُنسِّق وحسب الأعداد بل توضِّح العملة الصحيحة كذلك. كما أن بعض البلدان
تستعمل فيها وحدات قياس مغايرة للمسافات والأوزان وما شابهها. تحقق من أن
البرمجية تتبع المعايير المناسبة للغتك وبلدك. أحيانا تحتاج للتحويل من
وحدة لأخرى في الترجمة مثل التحويل من ميل إلى كيلومتر، وستجد بعض المواقع
أو البرمجيات التي تساعدك في هذا، لكن إذا كانت الكمية مذكورة على سبيل
المثال ليس وحسب (عادة ما تكون عددا صحيحا) فمن الأفضل أن تترك العدد كما
هو وتغير الوحدة فقط، فمثلا إذا كانت لعبة تتحدث عن بعض الذهب وزنه 100
رطل فربما من المستحسن تحويلها إلى 100 كجم لتكون أسهل في القراءة، لكن
عليك التأكد من أنه لا ضرر من فعل هذا وفي حال الشك ترجمها حرفيا أو حول
القيمة.

\subsection{الصور}
الصُّور معينات عظيمة على التوضيح، لكن شخصين من ثقافتين مختلفتين قد
ينظران إلى الصورة ذاتها فيستنتجان أمرين مختلفين تماما. فالصورة قد تشير
إلى معالم جغرافية أو أنشطة معينة أو أسماء معينة إلخ تخص ثقافة بعينها
ولا يعرفها جمهور المستخدمين أو لا تحمل عندهم الدلالة المقصودة.

قد تمر عليك سنوات من الاستغال بالتوطين لا تحتاج فيها لتغيير صورة واحدة،
لكن من المهم معرفة أن بعض الصور قد لا تناسب الجمهور الذي تستهدفه، لذا
ضع في الحسبان أن بعض الصور قد تحمل معان سيئة في بعض الثقافات أو قد لا
توصل الرسالة المطلوبة.

أحيانا ما تحتوي الصور على نصوص تنبغي ترجمتها، ومع أن هذه لا تعد ممارسة
برمجية جيدة إلا أن الحاجة قد تدفعك إليها لسبب أو لآخر.

توجد برمجيات تحرير صور عديدة لكن من الأفضل أن تراجع مطوري المشروع لتعرف
ما الملفات التي ستستخدمها وهل تدعم البرمجية صورا بديلة في الترجمة أم
لا، بدلا من أن تضيع وقتك في العمل على صُور لن تُستعمل. احرص على مراجعة
الكتابات الظاهرة في الصور بعناية، فاحتمال الأخطاء الإملائية وغيرها أكبر
لكونك لا تحررها في أداة ترجمة متخصصة. كذلك اسأل عن الصور في أنساق تحرير
وسيطة عِوضا عن العمل على أنساق قد ينتج عنها تدهور في الجودة أو الذوق أو
تتطلب جهدا عظيما و مهارة كبيرة لترجمة الكتابات الظاهرة فيها.

\subsection{الترتيب والقوائم}
تختلف قواعد الترتيب الهجائي للكلمات من لغة لأخرى (و في العربية يوجد كذلك
الترتيب الأبجدي)، حتى اللغات التي تُكتَب بنفس نظام الكتابة (العربي أو
اللاتيني مثلا) تختلف القواعد فيما بينها بسبب الاختلافات في طبيعة كل
لغة، أو حتى لممارسات تاريخية. تحقق من المواضِع التي يرد في ترتيب
للكلمات وانظر هل تتبع القواعد المناسبة للغتك أم لا. عادة ما تؤخذ هذه
القواعد من المحلية أو من مكتبة برمجية.

\chapter{دراسات حالة}
نعرض في هذا القسم بعض دراسات الحالة لمشروعات حرة شهيرة، وكيفية تقييمها.
قد ترغب في مراجعة قسم “اختيار منتَج أو مشروع” في صفحة
\at[ref:36346325] إذ سنستعمل بعضا من المعايير المذكورة هناك
لتقييم هذه المشروعات.

\section[ref:34376426]{رسم تُكس}
“رسم تُكس” برمجية رسم للأطفال.

\startitemize[1]
\item ترجمته حوالي 1500 كلمة فقط.
\item إيداع الترجمات الجديدة بسيط للغاية؛ تكفي رسالة بريد إلكتروني إلى
المبرمج.
\item يحتوي على بعض الدعم المتقدم لعرض النصوص لكن على كل لغة التحقق من
تحقق متطلباتها الخاصة. يدعم اللغات التي تكتب من اليمين إلى اليسار
والعديد من اللغات الآسيوية بالإضافة إلى اللغات التي تتطلب عرضا مُعقَّدا
للنصوص.
\item يستخدم ملفات جتتكست (PO) العادية التي يدعمها فِرتال وأدوات ترجمة
أخرى عديدة.
\item موجه للمستخدمين النهائيين بخاصة الأطفال الذين أغلبهم لا يعرفون أية
لغات أجنبية.
\item عديد المنصات ومتوفر للعديد من أنظمة التشغيل.
\item يلزم إجراء بسيط لاختبار لغة الواجهة، لكن من السهل استبدال ملف أي
من اللغات الموجودة للاختبار، وتوجد برمجية تضبيط لاختيار اللغة بعد
إضافتها رسميا إلى البرمجية.
\item لا حاجة لترجمة أي برمجيات إضافية، لكن يمكن ترجمة الموقع وبعض
الوظائف الإضافية.
\stopitemize
من هذه القائمة يظهر جليا أن “رسم تُكس” مرشح جذاب للترجمة، والنص ليس
تقنيا للغاية وإن كان عليك التفكير في أسماء الألوان والأشكال، وهو مشروع
مثالي لشخص حديث عهد بالتوطين؛ فالعمل عليه ممتع ويظهر أثره بسهوله.
\CURL{http://www.tuxpaint.org}

\section{جتك+ و xdg-user-dirs}
جتك+ منصة لبناء التطبيقات ومن أشهر ما بُني عليها مشروع گنوم. توفر جتك+
للبرمجيات بنية تحتية شائعة: مربعات حِوار لفتح الملفات، وخيارات الطباعة،
والأهم - فيما يخص هذا القسم من الكتاب - بعض المقاطع الشائعة مثل “OK” و
“Cancel” و “Preferences” و “Copy” و “Paste” إلخ. ولأن كثيرا من
البرمجيات تستعمل هذه المقاطع من جتك+ فترجمتها تفيد كل هذه البرمجيات. و
مع أن ترجمة جتك+ كبيرة نوعا ما إلا أنه من السهل التعرف على المقاطع
المهمة بمجرد النظر، أو بالبحث عن العناصر المؤشر عليها أنها “stock
items”.

ويوجد مشروع اسمه xdg-user-dirs له ملف ترجمة صغير فيه أسماء الأدلة
الشائعة في حساب أي مستخدم؛ “Desktop” و “Documents” و “Music” و
“Pictures”، وهذه الأدلة تكثر رؤيتها عند فتح ملف أو حفظه مما يعظم
الفائدة وراء ترجمتها مع بساطة المجهود المبذول.

لنكرر تقييمنا بناء على أسئلتنا السابقة:

\startitemize[1]
\item تحتوي جتك+ على كلمات عديدة لكن العناصر المهمة حوالي 100 كلمة، و
xdg-user-dirs أقل من 50 كلمة.
\item تُودع ترجمات جتك+ في مشروع گنوم بسهولة ويوجد أكثر من أسلوب مقبول
لإيداع الترجمات. تُودع ترجمات xdg-user-dirs إلى “مشروع الترجمة” وهذا
أصعب قليلا، لكن قد تجد من يساعدك في هذا الأمر.
\item على الأغلب لن توجد اعتبارات خاصة تتطلبها لغتك.
\item يستخدم ملفات جتتكست (PO) العادية التي يدعمها فِرتال وأدوات ترجمة
أخرى عديدة.
\item تُستخدم جتك+ في برمجيات تعمل على أنظمة تشغيل كثيرة، بينما
xdg-user-dirs لا تستخدم غالبا إلا على أنظمة لينُكس.
\item غالبا ما يعتمد تغيير لغة الترجمة على خيار اللغة في النظام.
\item لا تعتمد xdg-user-dirs على أي مكتبات خارجية. أجزاء صفيرة من ترجمة
جتك+ تأتي من Glib لكنها ليست مهمة في البداية غالبا، ويمكن ترجمة Glib
عبر مشروع گنوم أيضا.
\stopitemize

لا يحتاج هذان المشروعان إلا إلى قدر صغير من العمل لكن تأثيرهما كبير
ويظهر في البرمجيات الكثيرة التي تستخدمهما. يظهر أثر ترجمة جتك+ على نحو
أبرز في سطح المكتب گنوم و تطبيقاته حيث أغلب البرمجيات تستخدم ولو نزرا
يسيرا من هذه المقاطع.

\section{فَيَرفُكس}
فَيَرفُكس متصفح وِب شهير ينتجه مشروع موزيلا.

\startitemize[1]
\item فَيَرفُكس مشروع ضخم تتجاوز نصوصه عشرين ألف كلمة للتطبيق وحده،
بالإضافة إلى العديد من الأمور الهامة الأخرى قبل أن تصبح الترجمة معتمدة
رسميا، كما أن الترجمات الجزئية لا تُقبل.
\item إيداع ترجمة مُستحدثة في مشروع موزيلا أمر بالغ الصعوبة، إذ يتطلب
العديد من الخطوات التقنية بعضها يشمل مفاتيح تعمية ونظم إدارة إصدارات،
ولا تقبُل إلا الترجمات التامة.
\item يدعم فَيَرفُكس العديد من اللغات بما فيها تلك التي تتطلب خصائص
متقدمة لعرض النصوص، كما يمكن حَزم خطوط خاصة مع النسخ المُوَظَّنة.
\item تحتوي عدة الترجمة على مُحوَّلات تتيح ترجمة برمجيات موزيلا باستخدام
ملفات PO، لكن مشروع موزيلا لا يدعم هذه الأدوات وبالتالي على فرق الترجمة
تولي الأمر بأنفسهم. توجد برمجيات أخرى لكن لا شيء مما يعتاد المترجمون
استخدامه في المشروعات الأخرى.
\item فَيَرفُكس موجه للمستخدم النهائي كما أن شعبيته كبيرة.
\item عديد المنصات ومتوفر للعديد من أنظمة التشغيل.
\item توطينات فَيَرفُكس الرسمية متاحة للتنزيل لكل لغة على حدة، ويمكن
تنزيل حزم اللغات وتنصيبها لكن يتطلب هذا ملحقات إضافية لتفعيل لغة
الواجهة الجديدة.
\item لا حاجة لترجمة أي برمجيات أخرى، لكن توجد حاجة لترجمة العديد من
صفحات المنتجات والمواد التسويقية.
\stopitemize

لفَيَرفُكس شعبية كبيرة ويشكل هدفا جيدا لفريق ترجمة مُتمرِّس. لغة النص
تقنية في بعض المواضع، كما أنه مشروع ضخم بلا إثابات أثناء العمل عليه،
لكنه استثمار ناجح لمن يقدر عليه لشعبيته الكبيرة، كما ستستفيد من دعاية
موزيلا في الترويج لعملك.


\section{مُنصِّب دِبيان}
دِبيان من مشروعات البرمجيات الحرة الكبيرة، إذ يبني واحدة من أشهر توزيعات
لينُكس، وهي أيضا الأساس لتوزيعة أوبونتو. قد تكون أداة التنصيب أول شيء
يراه مستخدم دبيان أو أوبونتو، وهو مشروع جيد التنظيم ويبدي اهتماما كبيرا
لدعم اللغات المختلفة.

\startitemize[1]
\item ليس من الضروري ترجمة المنصِّب كله لكن يوجد ملفان يجب أن تتم
ترجمتهما بالكامل، وفيهما أكثر من عشرة آلاف كلمة.
\item يرحب المشروع بالمساهمات ويقبلها بأكثر من وسيلة من ضمنها استخدام
خادوم بُوتِل.
\item يدعم مُنصِّب دبيان لغات عديدة بما فيها تلك التي تتطلب العرض
المُعقَّد للنصوص، كما يمكن حزم خطوط خاصة مع الإصدارات المُوَطَّنة.
\item يستخدم ملفات جتتكست (PO) التي يدعمها فِرتال وكثير من أدوات الترجمة
الأخرى، كما يستضيفون خادوم بوتل.
\item لا يُعد المُنصِّب من البرمجيات الموجهة للمستخدم العادي وغالبا لا
يستخدمه غير أشخاص لهم خلفية تقنية، كما أنه لا يُستخدم كثيرا.
\item لا يُستخدم سوى في توزيعات لينُكس بعينها.
\item بمجرد أن تصبح الترجمة رسمية يستطيع المستخدم اختيارها بيسر من قائمة
اللغات، لكن قد تكون الاختبارات صعبا قليلا.
\item من الضروري ترجمة أسماء بعض البلاد إضافة إلى بعض الملفات الأخرى
لإتمام ترجمة المُنصِّب، لكنها ليست متطلبات مُلزِمة.
\stopitemize
قد يكون مُنصِّب دبيان هدفا جيدا للمترجمين ذوي الخلفية التقنية لكنه لن
يصل إلى كثير من المستخدمين، كما تغلب اللغة التقنية على النص. عتبة
الدخول مرتفعة وتتطلب ترجمة الكثير من المقاطع قليلة الأثر (مثل المناطق
الزمنية). المشروع جيد التنظيم ويقدّر الترجمة، كما قد تستقي منه ترجمات
مشروعات أخرى كما أوضحنا سلفا. إذا كان لديك فريق واسع الموارد أو طويل
الأمد فقد ترغب في العمل عليه.

\stopbodymatter

\stoptext
